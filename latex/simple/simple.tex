\documentclass{lab}

% For a real lab don't use blindtext
\usepackage{blindtext}

\labTitle{Laboratory Report \#1 --- My First Lab Report}
\labAuthor{Thomas Gregor}
\labCollaborators{Joshua Shaevitz, Peter Andolfatto}
\labDate{Monday, November 14}

\begin{document}

\maketitle

\section{Introductory Questions}
\label{Sec:Intro_Questions}

\begin{enumerate}

\item \textbf{You usually have to answer introdcutory questions this is
    a nice way to lay that out.}

  Woah, this is so nice! Here is an equation
  \begin{equation}
    \int_a^b f(x) \dd{x} = F(a) - F(b)
    \label{Eq:cool_eq}
  \end{equation}

\item \textbf{Is it true that they are numbered automatically?}

  It appears so! I thought Eq.~\eqref{Eq:cool_eq} was so interesting and
  fun I want to talk about it again over here.

\end{enumerate}

\section{Introduction}
\label{Sec:Intro}

\textbf{Focus Sentence: } We attempt to measure the length of the
flagella of \textit{E. Coli} using only a ruler, a microscope, and
tweezers.

We used five strains of \textit{E. coli} whose characteristics are
summarized in Table~\ref{table:strains}: some have flagella that we can
measure, but some are also too unfriendly to aproach safely.

\begin{table}[h]
  \centering
  \begin{tabular}{rll}
    \toprule
    Strain \# & Has Flagella & Is Friendly \\
    \midrule
    1         & no           & no \\
    2         & yes          & \textbf{no} \\
    3         & no           & yes \\
    4         & yes          & yes \\
    \bottomrule
  \end{tabular}
  \caption{Summary of Strains}\label{table:strains}
\end{table}

We also have some beautiful figures such as Fig.~\ref{Fig:1}, of which
Fig.~\ref{Fig:b_bright} is clearly the best.

\blindtext[2]

\begin{figure}[ht]
  \centering
  \begin{subfigure}{0.45\textwidth}
    \includegraphics[width=\textwidth,height=2in]{smile}
    \caption{Happy}\label{Fig:b_bright}
  \end{subfigure}
  \hfill
  \begin{subfigure}{0.45\textwidth}
    \includegraphics[width=\textwidth,height=2in]{smile}
    \caption{Smiley}\label{Fig:b_dark}
  \end{subfigure}
  \\
  \begin{subfigure}{0.45\textwidth}
    \includegraphics[width=\textwidth,height=2in]{smile}
    \caption{Yellow}\label{Fig:d_bright}
  \end{subfigure}
  \hfill
  \begin{subfigure}{0.45\textwidth}
    \includegraphics[width=\textwidth,height=2in]{smile}
    \caption{Cheerful}\label{Fig:d_dark}
  \end{subfigure}
  \caption{Here are four figures in a square pattern.}\label{Fig:1}
\end{figure}

\blindtext[2]

\begin{figure}[ht]
  \centering
  \includegraphics[width=0.4\textwidth]{smile}
  \caption{Here is a figure all by itself.}\label{Fig:gel}
\end{figure}

\blindtext[2]

\section{Concluding Questions}
\label{Sec:Con_Questions}

\begin{enumerate}
\item \textbf{Describe some other things that we can do here with math.}
  \begin{enumerate}
  \item Easy differentials with nice spacing!
    \begin{align}
      \dd{m} &= \lambda \dd{x} \\
      \dv{m}{x} &= \lambda \\
      \dv{x}\qty\Big[V\ell^2] &= \lambda
    \end{align}

  \item Nice brackets and parentheses that resize!
    \begin{equation}
      \qty(\frac{1}{1 + x})^2 = \qty[1 + x + x^2 + x^3 + \cdots + x^n]^2
    \end{equation}

  \item We've got all sorts of vectors!
    \begin{equation}
      \dd{\va*{x}} = \va{a} \vdot \va*{a} \vdot \vb{b} \cross \vb*{b} \cross \vu{u} \cross \vu*{u}
    \end{equation} 

  \item We've got partials!
    \begin{equation}
      \pdv{y}{x}
    \end{equation}
  \end{enumerate}
  
  For more information see http://mirrors.ibiblio.org/CTAN/macros/latex/contrib/physics/physics.pdf.

\item \textbf{What else can we do?}

  For chemistry, we can say that there \ce{O2 + 2H2 -> 2H2O}, or that
  \(\conc{H2O} = \SI{100}{\micro\molar}\) is increasing.

  We can also make differential equations:
  \begin{equation}
    \dv{\conc{A}}{t} = -k\conc{A}
  \end{equation}

  For units, we can say that magnetic fields are measured in
  \si{\tesla}, $\SI{1}{\newton} =
  \SI{1}{\kilo\gram\meter\per\second\squared}$, and that the speed of
  light is \SI{3e9}{\meter\per\second}, and that human speeds are
  usually \SIrange{0}{30}{\meter\per\second}.
\end{enumerate}

\section{Acknowledgments}
\label{Sec:Ack}

\pledge{Jake Waksbaum}

\section{MATLAB Code}
\label{Sec:MATLAB}

\lstinputlisting[language=Matlab]{simple.m}

\end{document}
