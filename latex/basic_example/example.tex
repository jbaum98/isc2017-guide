\documentclass{article}

\usepackage{amsmath}
\usepackage{amssymb}

\title{ISC 2021 \LaTeX\ Basics}
\author{ISC 2020}

\begin{document}

%%%%%%%%%%%%%%%%%%%%%%%%%%%%%%%%%%%%%%%%%%%%%%%%%%%%%%%%%%%%%%%%%%%%%%%%%%%%%%%%

\maketitle

This document shows off some of the features of \LaTeX. First, to display words
in the output PDF, we simply type them into the body (the space between
\verb!\begin{document} \end{document}!).
Now, if we put a single newline between two pieces of text, it has no effect!
The text is simply concatenated together as if we continued on the same line.

However, with, two newlines, we do see an effect. Two newlines starts
a new paragraph, which \LaTeX\ automatically indents.\\
This can alternatively be done with a double backslash (\verb!\\!), but we don't
get the indentation. \par
Or, with the command \verb!\par! (for paragraph!), we get the newline and the
indentation. \par

Notice how the commands we've been using all involve the backslash (\verb!\!)?
That's because the backslash is a \emph{reserved character}, just like the
semicolon in java. The backslash will begin every command in \LaTeX. 
Arguments of the commands will normally be surrounded by curly braces. \par

To do more formatting, we can use commands. It's not like Microsoft Word,
where what you see is what you get (wysiwyg). We can't just turn on italics.
Instead, we use the command \verb!\textit!. \textit{This text is italicized!}.
\textbf{This text is bold face, using the command} \verb!\textbf!. If we just
want to emphasize some text, we can use \verb!\emph!. This command makes the
\emph{text stand out} from the surrounding text. \textit{So, if we're in
  italics, and we want \emph{some text to stand out},} \verb!\emph!
\textit{will de-italicize it!} \textbf{Similarly, in bold face, text that
  \emph{stands out} will be italicized.}\par

In addition to commands, \LaTeX\ has structures called \emph{environments}.
Environments can do a lot of things:
\begin{enumerate}
\item Like make this numbered list!
\item Or make it really easy to typeset math (perhaps the most
  distinctive feature of \LaTeX) like $a^n + b^n \neq c^n\ \forall\ n \geq 3$.
\item Or add plots and pictures with captions (examples of which you can see in
  the sample labs).
\end{enumerate}
Notice that we use the command \verb!\item! in enumerate to create another
numbered point. If you want an unlabeled list, you can use the environment
\verb!itemize!. It too uses the \verb!\item! command. \par

Back to the math mode we were describing earlier. There, we used a shorthand
to enter \emph{inline math mode}--single dollar signs on either side of the
equations causes \LaTeX\ to enter math mode. If we use double dollar signs,
we can enter the display math environment which puts math on a separate
line:
$$ \forall\ x \in \mathbb{Z},\; x\ \vert\ 0.$$
There are other more advanced math modes that exist (look at the amsmath package
and its align environment for starters).\par

One last environment is worth explicit mention here. The \emph{verbatim}
environment is the easiest way for you to enter your \textsc{MATLAB} code, which
you will be required to do for your labs. Here's an example:

\begin{verbatim}
This environment uses a different font that is monospaced (makes code 
easier to read). It also gets rid of all protected characters, 
so you can backslash \\ and curly brace {} freely. 

Even more importantly, the percent sign (%), which is used to
comment lines in both MATLAB and LaTeX, no longer has any power.
We can also start newlines by inserting a newline--this environment
is much more wysiwyg. 

However, that leads to pitfalls, like lines that run off the page page page page page page page page page
\end{verbatim}

Back at the beginning of this document, we talked about the body of the document.
The space before the \verb!\begin{document}! is known as the preamble---in it,
you can define commands, tell \LaTeX\ what packages to use and insert the title.
However, to make that title display, we must put the command \verb!\maketitle!
at the beginning of the body. At the very beginning of the document, we use the
\verb!\documentclass! command, which tells \LaTeX\ how to format the whole file.
In almost every case, you will use the article class. If you decide to use
Jake Waksbaum's lab document class, you can replace the argument \verb!article!
with \verb!lab!.

One quick note about a command that has been scattered throughout this document:
the \verb!\verb! command enters an inline \emph{verbatim} mode. To enter the
arguments, we use exclamation points (!) instead of curly braces ({}), since
the curly brace is no longer protected. The author of the document suspects you
will \emph{NEVER} need to use this command. To enter code, use the verbatim
environment.

%%%%%%%%%%%%%%%%%%%%%%%%%%%%%%%%%%%%%%%%%%%%%%%%%%%%%%%%%%%%%%%%%%%%%%%%%%%%%%%%

\end{document}