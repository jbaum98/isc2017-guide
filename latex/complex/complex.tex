\documentclass{lab}

\labTitle{Lab \#11: Macroscopic Methods for Measuring Planck's Constant}
\labAuthor{Jake Waksbaum and Oriel Farajun}
\labDate{Friday, May 12}

\addbibresource{complex.bib}

\begin{document}

\maketitle

\begin{abstract}
  In this paper we present two methods for measuring Planck's constant $h$. The
  first, based on earlier work~\parencite{zhou}, measures the stopping voltage
  of LEDs to determine the energy of the photons produced and fits the result
  from many LEDs to the equation $E = h\nu$. The second is novel, and involves
  measuring the radiation spectrum of the sun using a photodiode and light
  filters, and fitting it to the blackbody radiation curve predicted by Planck's
  Law. We measured $h$ to be \num{6.2781 +- .12064 e-34} and \num{7.7868 +-
    .60495 e-34} using the two methods respectively.
\end{abstract}

\section{Introduction}
\label{Sec:Intro}

Since it's development in the beginning of the 20th century, quantum mechanics
has revolutionized physics by overturning widely held beliefs about the nature
of the universe and solving previously intractable problems. At the same time,
it shifted the focus of physics from the realm of common experience to the
quantum world for which most people have no intuition. Our goal was to measure
Planck's constant, one of the fundamental constants of quantum mechanics, by
observing familiar phenomena in the macroscopic world.

One of the problems that originally motivated the development of quantum
mechanics was the paradox of blackbody radiation. Classical mechanics predicted
a blackbody would radiate infinite energy. Planck introduced the idea of
intensity of light of wavelength $\lambda$ radiated by a blackbody at some temperature
$T$ that became known as Planck's Law:
\begin{equation}
  I_T(\lambda) = \frac{2hc^2}{\lambda^5} \frac{1}{\exp(\frac{hc}{k_B T} \frac{1}{\lambda})-1}
\end{equation}
where $k_B = \num{1.38064852e-23}$ is Boltzmann's constant.

Another one of the early successes of quantum mechanics was the explanation of
the photoelectric effect due to Einstein. The photoelectric effect is the
production of a current due to incident light, and according to classical
mechanics the kinetic energy of the electrons should increase with the intensity
of the incident light. However, experiments showed that the kinetic energy of
the electrons depended on the frequency of the incident light. Einstein
explained this phenomenon using the idea of light as discrete quanta called
photons, and derived the Planck-Einstein relation for the energy of a photon:
\begin{equation}
  E = h\nu
\end{equation}
where $h$ is the same constant introduced by Planck in explaining blackbody
radiation.

Our first method for measuring Planck's constant is related to the photoelectric
effect. An LED produces light by a mechanism that is almost the reverse of the
photoelectric effect: electrons travel across a voltage drop and emit photons of
a specific frequency following the Planck-Einstein relation. The stopping
voltage of an LED is the lowest voltage at which it produces light, and at that
voltage the voltage drop across the LED is equal to the voltage drop of the
electron and the emitted photon. Therefore we have
\begin{equation}
  E = q_e V_{\text{stop}} = h\nu
\end{equation}
where $q_e = \SI{1.60217662e-19}{\coulomb}$ is the charge on an electron and $\nu$
is the frequency.

Determining when an LED just begins to produce light is subjective and
error-prone. Instead, one can use an RC circuit in which a capacitor discharges
to power an LED (Fig.~\ref{fig:discharging}). Assuming that the voltage drop
across the LED is constant, the voltage drop across the capacitor will be given
by
\begin{equation}
  V(t) = \qty(\frac{Q_0}{C} - V_{\text{stop}}) \exp(-\frac{t}{RC}) + V_{\text{stop}}
  \label{eq:led_cap}
\end{equation}
Fitting voltage data to the form $V = A\exp(Bt) + C$ allows us to determine the
stopping voltage, and fitting the stopping voltages versus the wavelengths of
the LEDs using the Planck-Einstein relation allows us to determine $h$.

Our second method is related to blackbody radiation. The sun is approximately a
blackbody radiating at \SI{5778}{\kelvin}. We attempt to measure its radiation
spectrum using a photodiode circuit (Fig.~\ref{fig:photodiode}) that produces
voltage in response to the intensity of incident light. By covering the
photodiode with filters that admit only certain bands of light, we can measure
the relative intensity of light at different frequencies. Fitting this curve to
the curve predicted by Planck's Law allows us to determine $h$.

\section{Materials and Methods}
\label{Sec:MatsMeths}

The circuit shown in Fig.~\ref{fig:circuit} was used to first charge the
capacitor (Fig.~\ref{fig:charging}) and then to discharge the capacitor through
the LED (Fig.~\ref{fig:discharging}) for approximately $15\tau =
\SI{150}{\second}$. The voltmeter was connected to a computer that recorded the
voltage drop every \SI{0.01}{\second}. This process was repeated with all of the
LEDs available.

\begin{figure}[h]
  \centering
  \begin{subfigure}{0.45\textwidth}
    \centering
    \input{figs/charging.tex}
    \caption{Charging\label{fig:charging}}
  \end{subfigure}
  \hfill
  \begin{subfigure}{0.45\textwidth}
    \centering
    \begin{circuitikz}[scale = 0.7, transform shape]
  \draw (0,0) to[capacitor=0.01<\farad>] ++(0,-2)
  to[resistor=1<\kilo\ohm>] ++(0,-2)
  -- ++(-2,0)
  to[battery=10<\volt>] ++(0,4) to[short,-o]++(0,0);
  \draw (0,0) -- ++(2,0) to[led] ++(0,-4) -| (0,-4);
  \draw (0,-0.5) -- ++(-1,0) to[voltmeter] ++(0,-1) -- ++(1,0);
\end{circuitikz}
    \caption{Discharging\label{fig:discharging}}
  \end{subfigure}
  \caption{LED RC Circuit\label{fig:circuit}}
\end{figure}

The circuit shown in Fig.~\ref{fig:photodiode} was connected to a multimeter
measuring the voltage difference between $V_{\text{out}}$ and ground. The
\SI{+-15}{\volt} voltages were altered from \SIrange{9}{27}{\volt} as the
photodiode's generated voltage reached the upper bound determined by the input
voltage.

\begin{figure}[h]
  \centering
  \begin{circuitikz}[scale = 0.7, transform shape]
  \draw (0,0) node[op amp] (opamp) {};
  \draw (opamp.+) -| ++ (-0.5,-0.5) node[ground]{};
  \draw (opamp.-)++(-2.2,0) node[left]{$-\SI{15}{\volt}$} to[photodiode,l^=\raisebox{10pt}{Photodiode}] ++ (2,0) to[short,n=J] ++(0.2,0) -- (opamp.-);
  \draw (opamp.out) -- ++(2,0) to[short,n=out] ++(0.5,0) node[right]{$V_{\text{out}}$} ;
  \draw (J) -- ++(0,1.5) to[short,n=L] ++(0,0.7) -- ++(0,1) to[capacitor,l^=0.1<\micro\farad>]
  ++(4.5,0) -- ++ (0,-1) to[short,n=R] ++(0,-0.7) -- ++(0,1.5) |- (out);
  \draw (L) to[R,l^=75<\kilo\ohm>] (R);
  \draw (opamp.down) to[short,n=B] ++(0,-1) node[below]{$+\SI{15}{\volt}$};
  \draw (B) to[capacitor,l_=0.1<\micro\farad>] ++(2.5,0) node[ground]{};
  \draw (opamp.up) to[short,n=T] ++(0,1) node[above]{$-\SI{15}{\volt}$};
  \draw (T) to[capacitor,l^=0.1<\micro\farad>] ++(2.5,0) node[ground]{};
\end{circuitikz}
  \caption{Photodiode Circuit\label{fig:photodiode}}
\end{figure}

The photodiode was covered with various filters and exposed to the sun's light,
and the generated voltage was recorded. The date, time, and latitude and
longitude were also recorded for each trial.

The photodiode produces voltage more readily in response to light of different
frequencies; to correct for this, the spectral response curve for the photodiode
was obtained from the manufacturer and the intensities were divided by the
corresponding responsivity.

The atmosphere itself absorbs light of different wavelengths in different ways.
To correct for this, we followed the recommendations of~\cite{nasa}. We used the
date, time, and location to calculate the air mass using formulas from
\cite{noaa}. We used Ångstrom turbidity coefficients $\alpha = 1.3$ and $\beta = 0.085$.

\section{Results and Discussion}
\label{Sec:ResDisc}

Fig.~\ref{fig:complex} shows the change in voltage over time for a specific
trial. When attempting to fit the data to a decay curve, we first tried a
fitting it to a single decay $V = A\exp(Bt) + C$ as predicted by theory.
However, we noticed that in-between charging and discharging, the capacitor
experienced some voltage leakage, decaying even though it was not attached to
any circuit. In addition, plotting the residual yielded a non-random pattern
indicating that the fit did not capture some underlying phenomenon occurring in
the circuit.

This motivated us to try a fit with two exponentials to account for the leakage
in the form $V = A\exp(Bt) + C\exp(Dt)$. We expected $D$ to be very small to
correspond to the very slow decay of the leakage. We took $C$ to be the stopping
voltage because this is the constant voltage other than the main decay that is
present before the leakage; in this way we try to ignore the leakage. However,
Fig.~\ref{fig:complex_resid} shows that the residual is still non-random,
similar to the residual with the single exponential fit. This may be due to a
non-constant voltage across the LED or to other phenomena we have not
considered.

\begin{figure}[h]
  \centering
  \begin{subfigure}{0.48\textwidth}
    \centering
    % This file was created by matlab2tikz.
%
\definecolor{mycolor1}{rgb}{0.00000,0.44700,0.74100}%
\definecolor{mycolor2}{rgb}{0.85000,0.32500,0.09800}%
\definecolor{mycolor3}{rgb}{0.92900,0.69400,0.12500}%
\definecolor{mycolor4}{rgb}{0.49400,0.18400,0.55600}%
%
\begin{tikzpicture}[%
trim axis left
]

\begin{axis}[%
width=1.902in,
height=2in,
at={(0in,0in)},
scale only axis,
xmin=0,
xmax=180,
xlabel style={font=\color{white!15!black}},
xlabel={Time (\si{\second})},
ymin=0,
ymax=12,
ylabel style={font=\color{white!15!black}},
ylabel={Voltage (\si{\volt})},
axis background/.style={fill=white},
axis x line*=bottom,
axis y line*=left,
legend style={legend cell align=left, align=left, draw=white!15!black},
legend style={font=\small}
]
\addplot [color=mycolor1, line width=2.0pt]
  table[row sep=crcr]{%
0	10.476\\
0.0800000000000001	10.425\\
0.15	10.38\\
0.34	10.263\\
0.38	10.239\\
0.86	9.954\\
0.899999999999999	9.931\\
1.01	9.868\\
1.06	9.84\\
1.17	9.777\\
1.23	9.744\\
1.31	9.699\\
1.38	9.661\\
1.44	9.628\\
1.5	9.595\\
1.55	9.568\\
1.64	9.52\\
1.7	9.487\\
1.81	9.429\\
1.86	9.402\\
2.01	9.324\\
2.06	9.297\\
2.09	9.282\\
3.039	8.811\\
3.079	8.792\\
3.189	8.74\\
3.239	8.717\\
3.339	8.67\\
3.389	8.647\\
3.469	8.61\\
3.539	8.579\\
3.619	8.542\\
3.689	8.511\\
3.769	8.474\\
3.789	8.466\\
3.849	8.439\\
3.929	8.404\\
3.989	8.377\\
4.099	8.33\\
4.159	8.303\\
4.279	8.252\\
4.319	8.235\\
4.449	8.18\\
4.489	8.163\\
4.679	8.084\\
4.719	8.067\\
4.749	8.055\\
5.639	7.702\\
5.679	7.687\\
5.849	7.622\\
5.899	7.604\\
6.009	7.563\\
6.059	7.545\\
6.169	7.504\\
6.219	7.486\\
6.319	7.449\\
6.379	7.428\\
6.459	7.399\\
6.529	7.375\\
6.609	7.346\\
6.679	7.322\\
6.739	7.301\\
6.799	7.28\\
6.869	7.255\\
6.959	7.225\\
7.019	7.204\\
7.109	7.174\\
7.169	7.153\\
7.189	7.147\\
7.239	7.13\\
7.329	7.1\\
7.379	7.083\\
7.499	7.044\\
7.549	7.027\\
7.689	6.982\\
7.729	6.969\\
7.909	6.912\\
7.949	6.899\\
8.179	6.827\\
8.209	6.817\\
8.249	6.806\\
8.289	6.793\\
9.179	6.529\\
9.219	6.518\\
9.439	6.455\\
9.489	6.442\\
9.649	6.397\\
9.699	6.384\\
9.839	6.345\\
9.889	6.332\\
10.009	6.299\\
10.059	6.286\\
10.159	6.259\\
10.219	6.244\\
10.299	6.223\\
10.349	6.21\\
10.429	6.189\\
10.489	6.174\\
10.569	6.153\\
10.629	6.138\\
10.709	6.117\\
10.779	6.1\\
10.849	6.082\\
10.919	6.065\\
10.979	6.05\\
11.059	6.031\\
11.129	6.013\\
11.219	5.992\\
11.279	5.977\\
11.369	5.956\\
11.439	5.938\\
11.559	5.911\\
11.619	5.896\\
11.719	5.873\\
11.769	5.861\\
11.869	5.838\\
11.919	5.826\\
12.059	5.795\\
12.109	5.783\\
12.259	5.75\\
12.299	5.74\\
12.329	5.734\\
12.369	5.725\\
12.539	5.688\\
12.579	5.679\\
12.779	5.636\\
12.819	5.627\\
13.189	5.55\\
13.229	5.541\\
14.359	5.318\\
14.399	5.311\\
14.659	5.262\\
14.709	5.254\\
14.919	5.215\\
14.959	5.208\\
15.109	5.181\\
15.149	5.174\\
15.259	5.155\\
15.299	5.148\\
15.439	5.123\\
15.479	5.116\\
15.589	5.097\\
15.649	5.088\\
15.779	5.065\\
15.829	5.057\\
15.939	5.038\\
15.999	5.029\\
16.119	5.008\\
16.179	4.999\\
16.259	4.986\\
16.309	4.978\\
16.389	4.965\\
16.439	4.957\\
16.529	4.942\\
16.599	4.932\\
16.679	4.919\\
16.749	4.909\\
16.849	4.892\\
16.919	4.882\\
16.989	4.871\\
17.059	4.861\\
17.139	4.848\\
17.209	4.838\\
17.279	4.827\\
17.349	4.817\\
17.419	4.806\\
17.499	4.795\\
17.569	4.784\\
17.639	4.774\\
17.689	4.767\\
17.759	4.757\\
17.819	4.748\\
17.909	4.736\\
17.979	4.725\\
18.069	4.713\\
18.119	4.706\\
18.189	4.696\\
18.239	4.689\\
18.329	4.677\\
18.379	4.67\\
18.449	4.66\\
18.499	4.653\\
18.609	4.639\\
18.659	4.632\\
18.759	4.619\\
18.819	4.61\\
18.939	4.595\\
18.989	4.588\\
19.129	4.571\\
19.179	4.564\\
19.279	4.551\\
19.329	4.544\\
19.459	4.528\\
19.499	4.523\\
19.619	4.508\\
19.659	4.503\\
19.809	4.485\\
19.859	4.478\\
20.039	4.457\\
20.079	4.452\\
20.239	4.433\\
20.279	4.428\\
20.469	4.406\\
20.499	4.402\\
20.529	4.399\\
20.569	4.394\\
20.789	4.369\\
20.819	4.366\\
21.069	4.338\\
21.109	4.333\\
21.459	4.295\\
21.489	4.292\\
22.029	4.235\\
22.059	4.232\\
23.428	4.098\\
23.468	4.095\\
23.828	4.062\\
23.868	4.059\\
24.178	4.031\\
24.218	4.028\\
24.488	4.004\\
24.538	4.001\\
24.788	3.979\\
24.838	3.976\\
25.078	3.955\\
25.118	3.952\\
25.268	3.94\\
25.308	3.937\\
25.448	3.926\\
25.488	3.923\\
25.648	3.91\\
25.688	3.907\\
25.818	3.897\\
25.858	3.894\\
25.998	3.883\\
26.048	3.88\\
26.188	3.869\\
26.238	3.866\\
26.378	3.855\\
26.418	3.852\\
26.538	3.843\\
26.588	3.84\\
26.708	3.831\\
26.758	3.828\\
26.878	3.819\\
26.928	3.816\\
27.048	3.807\\
27.098	3.804\\
27.218	3.795\\
27.268	3.792\\
27.378	3.784\\
27.438	3.781\\
27.558	3.772\\
27.608	3.769\\
27.698	3.763\\
27.748	3.76\\
27.858	3.752\\
27.908	3.749\\
27.998	3.743\\
28.048	3.74\\
28.148	3.733\\
28.208	3.73\\
28.318	3.722\\
28.378	3.719\\
28.478	3.712\\
28.538	3.709\\
28.638	3.702\\
28.688	3.7\\
28.718	3.697\\
28.778	3.694\\
28.868	3.688\\
28.928	3.685\\
29.008	3.68\\
29.058	3.677\\
29.128	3.673\\
29.178	3.67\\
29.258	3.665\\
29.318	3.662\\
29.418	3.655\\
29.478	3.652\\
29.548	3.648\\
29.618	3.645\\
29.708	3.639\\
29.768	3.636\\
29.858	3.63\\
29.918	3.627\\
29.988	3.623\\
30.048	3.62\\
30.118	3.616\\
30.168	3.613\\
30.228	3.61\\
30.288	3.607\\
30.368	3.602\\
30.438	3.599\\
30.518	3.594\\
30.578	3.591\\
30.648	3.587\\
30.718	3.584\\
30.798	3.579\\
30.868	3.576\\
30.938	3.572\\
31.008	3.569\\
31.088	3.564\\
31.168	3.561\\
31.238	3.557\\
31.298	3.554\\
31.368	3.55\\
31.438	3.547\\
31.508	3.543\\
31.568	3.54\\
31.628	3.537\\
31.708	3.534\\
31.788	3.529\\
31.858	3.526\\
31.928	3.522\\
32.008	3.519\\
32.078	3.515\\
32.148	3.512\\
32.218	3.508\\
32.298	3.505\\
32.358	3.502\\
32.428	3.499\\
32.488	3.496\\
32.558	3.493\\
32.618	3.49\\
32.688	3.487\\
32.748	3.484\\
32.818	3.481\\
32.878	3.478\\
32.958	3.475\\
33.018	3.472\\
33.078	3.469\\
33.138	3.466\\
33.218	3.463\\
33.278	3.46\\
33.368	3.457\\
33.428	3.454\\
33.508	3.451\\
33.568	3.448\\
33.638	3.445\\
33.698	3.442\\
33.788	3.439\\
33.848	3.436\\
33.938	3.433\\
33.998	3.43\\
34.078	3.427\\
34.138	3.424\\
34.218	3.421\\
34.278	3.418\\
34.378	3.415\\
34.438	3.412\\
34.518	3.409\\
34.578	3.406\\
34.678	3.403\\
34.738	3.4\\
34.828	3.397\\
34.888	3.394\\
34.988	3.391\\
35.048	3.388\\
35.138	3.385\\
35.198	3.382\\
35.298	3.379\\
35.348	3.377\\
35.428	3.374\\
35.478	3.372\\
35.558	3.369\\
35.598	3.368\\
35.678	3.365\\
35.738	3.362\\
35.828	3.359\\
35.878	3.357\\
35.978	3.354\\
36.028	3.352\\
36.128	3.349\\
36.178	3.347\\
36.278	3.344\\
36.338	3.341\\
36.448	3.338\\
36.508	3.335\\
36.618	3.332\\
36.678	3.329\\
36.788	3.326\\
36.838	3.324\\
36.938	3.321\\
36.988	3.319\\
37.098	3.316\\
37.148	3.314\\
37.238	3.311\\
37.288	3.309\\
37.398	3.306\\
37.448	3.304\\
37.548	3.301\\
37.598	3.299\\
37.708	3.296\\
37.758	3.294\\
37.858	3.291\\
37.908	3.289\\
38.028	3.286\\
38.078	3.284\\
38.208	3.281\\
38.268	3.278\\
38.398	3.275\\
38.458	3.272\\
38.598	3.269\\
38.648	3.267\\
38.768	3.264\\
38.818	3.262\\
38.948	3.259\\
38.998	3.257\\
39.108	3.254\\
39.158	3.252\\
39.278	3.249\\
39.328	3.247\\
39.478	3.244\\
39.528	3.242\\
39.648	3.239\\
39.698	3.237\\
39.838	3.234\\
39.888	3.232\\
40.028	3.229\\
40.088	3.226\\
40.268	3.223\\
40.318	3.221\\
40.438	3.218\\
40.488	3.216\\
40.628	3.213\\
40.678	3.211\\
40.828	3.208\\
40.868	3.207\\
40.978	3.204\\
41.018	3.203\\
41.118	3.2\\
41.158	3.199\\
41.318	3.196\\
41.368	3.194\\
41.508	3.191\\
41.548	3.19\\
41.698	3.187\\
41.748	3.185\\
41.898	3.182\\
41.948	3.18\\
42.097	3.177\\
42.137	3.176\\
42.287	3.173\\
42.337	3.171\\
42.507	3.168\\
42.557	3.166\\
42.727	3.163\\
42.767	3.161\\
42.797	3.161\\
42.957	3.158\\
43.007	3.156\\
43.187	3.153\\
43.227	3.151\\
43.257	3.151\\
43.427	3.148\\
43.467	3.147\\
43.617	3.144\\
43.657	3.143\\
43.807	3.14\\
43.857	3.138\\
44.057	3.135\\
44.097	3.133\\
44.127	3.133\\
44.297	3.13\\
44.337	3.129\\
44.507	3.126\\
44.557	3.124\\
44.587	3.124\\
44.767	3.121\\
44.807	3.119\\
44.837	3.119\\
45.017	3.116\\
45.057	3.115\\
45.237	3.112\\
45.287	3.11\\
45.317	3.11\\
45.497	3.107\\
45.537	3.106\\
45.707	3.103\\
45.747	3.102\\
45.957	3.099\\
46.007	3.097\\
46.237	3.094\\
46.287	3.092\\
46.317	3.092\\
46.527	3.089\\
46.567	3.088\\
46.757	3.085\\
46.807	3.083\\
47.047	3.08\\
47.087	3.079\\
47.317	3.076\\
47.357	3.074\\
47.387	3.075\\
47.437	3.073\\
47.667	3.07\\
47.707	3.069\\
47.907	3.066\\
47.947	3.065\\
48.187	3.062\\
48.227	3.061\\
48.437	3.058\\
48.477	3.057\\
48.687	3.054\\
48.727	3.053\\
48.987	3.05\\
49.027	3.049\\
49.287	3.046\\
49.327	3.045\\
49.537	3.042\\
49.577	3.041\\
49.837	3.038\\
49.877	3.037\\
50.127	3.034\\
50.177	3.032\\
50.207	3.033\\
50.247	3.032\\
50.487	3.029\\
50.517	3.029\\
50.757	3.026\\
50.797	3.025\\
51.057	3.022\\
51.097	3.021\\
51.357	3.018\\
51.397	3.017\\
51.697	3.014\\
51.737	3.013\\
52.007	3.01\\
52.047	3.009\\
52.337	3.006\\
52.377	3.005\\
52.687	3.002\\
52.727	3.001\\
53.017	2.998\\
53.047	2.997\\
53.077	2.998\\
53.117	2.997\\
53.397	2.994\\
53.437	2.993\\
53.767	2.99\\
53.807	2.989\\
54.117	2.986\\
54.157	2.985\\
54.517	2.982\\
54.557	2.981\\
54.957	2.978\\
54.997	2.977\\
55.327	2.974\\
55.367	2.973\\
55.747	2.97\\
55.787	2.969\\
56.147	2.966\\
56.187	2.965\\
56.587	2.962\\
56.627	2.961\\
57.057	2.958\\
57.097	2.957\\
57.557	2.954\\
57.597	2.953\\
57.947	2.95\\
57.987	2.949\\
58.027	2.95\\
58.067	2.949\\
58.427	2.946\\
58.467	2.945\\
58.947	2.942\\
58.987	2.941\\
59.477	2.938\\
59.517	2.937\\
59.957	2.934\\
59.997	2.933\\
60.547	2.93\\
60.577	2.929\\
60.607	2.929\\
61.077	2.926\\
61.117	2.925\\
61.617	2.922\\
61.647	2.921\\
61.677	2.921\\
62.306	2.918\\
62.346	2.917\\
62.916	2.914\\
62.956	2.913\\
63.496	2.91\\
63.536	2.909\\
63.566	2.909\\
64.196	2.906\\
64.236	2.905\\
64.836	2.902\\
64.866	2.902\\
65.326	2.899\\
65.356	2.898\\
65.396	2.899\\
65.436	2.898\\
66.126	2.895\\
66.166	2.894\\
66.196	2.894\\
66.856	2.891\\
66.896	2.89\\
66.926	2.89\\
67.616	2.887\\
67.656	2.886\\
68.436	2.883\\
68.476	2.882\\
68.516	2.883\\
68.556	2.882\\
69.286	2.879\\
69.316	2.879\\
69.866	2.876\\
69.896	2.876\\
70.536	2.873\\
70.576	2.872\\
71.546	2.869\\
71.576	2.869\\
72.196	2.866\\
72.226	2.866\\
72.986	2.863\\
73.026	2.862\\
73.946	2.859\\
73.986	2.858\\
74.996	2.855\\
75.026	2.855\\
75.866	2.852\\
75.896	2.852\\
76.766	2.849\\
76.796	2.849\\
77.756	2.846\\
77.786	2.846\\
78.626	2.843\\
78.656	2.843\\
79.536	2.84\\
79.566	2.84\\
80.616	2.837\\
80.656	2.836\\
80.686	2.837\\
80.726	2.836\\
82.065	2.833\\
82.095	2.833\\
83.185	2.83\\
83.215	2.829\\
83.245	2.829\\
84.575	2.826\\
84.605	2.826\\
85.805	2.823\\
85.845	2.822\\
87.455	2.819\\
87.485	2.819\\
88.885	2.816\\
88.915	2.816\\
90.285	2.813\\
90.315	2.813\\
91.915	2.81\\
91.945	2.809\\
91.985	2.81\\
92.015	2.809\\
92.045	2.809\\
93.835	2.806\\
93.865	2.806\\
95.305	2.803\\
95.345	2.802\\
95.385	2.803\\
97.045	2.8\\
97.075	2.8\\
99.095	2.797\\
99.125	2.797\\
100.755	2.794\\
100.785	2.794\\
102.824	2.791\\
102.854	2.791\\
104.784	2.788\\
104.814	2.788\\
107.144	2.785\\
107.174	2.785\\
109.574	2.782\\
109.614	2.781\\
109.654	2.782\\
111.814	2.779\\
111.844	2.779\\
114.534	2.776\\
114.564	2.776\\
117.424	2.773\\
117.454	2.773\\
120.174	2.77\\
120.204	2.77\\
123.153	2.767\\
123.183	2.767\\
126.283	2.764\\
126.313	2.763\\
126.353	2.764\\
130.103	2.761\\
130.133	2.761\\
133.803	2.758\\
133.833	2.757\\
133.863	2.758\\
133.893	2.758\\
137.223	2.755\\
137.253	2.755\\
141.193	2.752\\
141.222	2.752\\
146.362	2.749\\
146.392	2.749\\
150.772	2.746\\
150.802	2.746\\
155.582	2.743\\
155.612	2.743\\
160.682	2.74\\
160.712	2.74\\
166.481	2.737\\
166.511	2.737\\
168.901	2.736\\
};
\addlegendentry{Experimental Data}

\addplot [color=mycolor2, dashed, line width=2.0pt]
  table[row sep=crcr]{%
0	10.4030593672708\\
0.50721021021021	10.1101620987329\\
1.01442042042042	9.82871833082581\\
1.52163063063063	9.55827905309084\\
2.02884084084084	9.29841285786936\\
2.53605105105105	9.04870525021059\\
3.04326126126126	8.80875798483349\\
3.55047147147147	8.57818842908208\\
4.05768168168168	8.35662895085521\\
4.56489189189189	8.14372633053162\\
5.0721021021021	7.93914119594963\\
5.57931231231231	7.74254747953762\\
6.08652252252252	7.55363189672695\\
6.59373273273273	7.37209344481292\\
7.10094294294294	7.19764292146214\\
7.60815315315315	7.03000246209616\\
8.11536336336336	6.86890509541128\\
8.62257357357358	6.71409431632355\\
9.12978378378378	6.56532367565584\\
9.63699399399399	6.42235638591068\\
10.1442042042042	6.2849649424981\\
10.8204844844845	6.11007332852866\\
11.4967647647648	5.9442121348028\\
12.173045045045	5.78691245748233\\
12.8493253253253	5.63772974141377\\
13.5256056056056	5.49624251577831\\
14.2018858858859	5.36205119539547\\
14.8781661661662	5.23477694427123\\
15.5544464464464	5.11406059815855\\
16.2307267267267	4.99956164306587\\
16.907007007007	4.89095724680842\\
17.7523573573574	4.76302622379428\\
18.5977077077077	4.64326528438416\\
19.4430580580581	4.53114764166279\\
20.2884084084084	4.42618047677984\\
21.1337587587588	4.3279027486337\\
21.9791091091091	4.23588314479049\\
22.9935295295295	4.13315406621222\\
24.00794994995	4.03820890435072\\
25.0223703703704	3.95044936535695\\
26.0367907907908	3.86932314630441\\
27.0512112112112	3.79432039985906\\
28.2347017017017	3.71393071860513\\
29.4181921921922	3.6405510845488\\
30.6016826826827	3.57355702986779\\
31.9542432432432	3.50408413515963\\
33.3068038038038	3.44144174502519\\
34.8284344344344	3.37827625805072\\
36.3500650650651	3.32200469178983\\
38.0407657657658	3.2666214585184\\
39.7314664664665	3.21784717426752\\
41.5912372372372	3.17084780811451\\
43.6200780780781	3.12644254582415\\
45.817988988989	3.08523031093901\\
48.18496996997	3.04759300729876\\
50.8900910910911	3.01165973529516\\
53.9333523523524	2.97853678814671\\
57.3147537537538	2.94887606189255\\
61.2033653653654	2.92186706995439\\
65.7682572572573	2.89728319964805\\
71.3475695695696	2.87449567571726\\
78.2794424424424	2.85334521341239\\
87.7473663663664	2.83177985684789\\
102.118322322322	2.80658477429206\\
130.353024024024	2.76520414964807\\
168.901	2.71224143162701\\
};
\addlegendentry{Fit}

\addplot [color=mycolor3, dashed, line width=2.0pt]
  table[row sep=crcr]{%
0	7.45212924422866\\
0.50721021021021	7.15997935737034\\
1.01442042042042	6.87928278185354\\
1.52163063063063	6.60959050726754\\
2.02884084084084	6.35047112600165\\
2.53605105105105	6.101510143153\\
3.04326126126126	5.86230931348844\\
3.55047147147147	5.63248600439991\\
4.05768168168168	5.41167258383412\\
4.56489189189189	5.19951583221769\\
5.0721021021021	4.99567637743678\\
5.57931231231231	4.79982815196764\\
6.08652252252252	4.61165787128944\\
6.59373273273273	4.43086453274531\\
7.10094294294294	4.25715893404967\\
7.60815315315315	4.09026321067186\\
8.11536336336336	3.92991039135596\\
8.62257357357358	3.7758439710658\\
9.12978378378378	3.62781750067201\\
9.63699399399399	3.48559419272485\\
10.1442042042042	3.34894654268212\\
10.8204844844845	3.17504636017779\\
11.4967647647648	3.01017626313156\\
12.173045045045	2.85386734781828\\
12.8493253253253	2.70567505919749\\
13.5256056056056	2.56517792656334\\
14.2018858858859	2.4319763648483\\
14.8781661661662	2.30569153817124\\
15.5544464464464	2.18596428239798\\
16.2307267267267	2.07245408364977\\
16.907007007007	1.96483810985465\\
17.7523573573574	1.83814214556839\\
18.5977077077077	1.71961574359157\\
19.4430580580581	1.60873211722895\\
20.2884084084084	1.50499844785011\\
21.1337587587588	1.40795369457331\\
21.9791091091091	1.31716654518442\\
22.9935295295295	1.21591572531275\\
24.00794994995	1.12244807345592\\
25.0223703703704	1.03616529614406\\
26.0367907907908	0.956515090829697\\
27.0512112112112	0.882987610557591\\
28.2347017017017	0.804318128838548\\
29.4181921921922	0.732657677914438\\
30.6016826826827	0.667381790563528\\
31.9542432432432	0.59987127259178\\
33.3068038038038	0.539189934111078\\
34.8284344344344	0.47822904747322\\
36.3500650650651	0.424160406896669\\
38.0407657657658	0.371222906278867\\
39.7314664664665	0.324892290523712\\
41.5912372372372	0.28057857769277\\
43.6200780780781	0.239100275614849\\
45.817988988989	0.201055569463626\\
48.18496996997	0.16682556458744\\
50.8900910910911	0.134781419645856\\
53.9333523523524	0.106027464873481\\
57.3147537537538	0.0812133932377132\\
61.2033653653654	0.0597679454795717\\
65.7682572572573	0.0417014209758681\\
71.3475695695696	0.0268593911642153\\
78.4485125125125	0.0153438961282681\\
88.0855065065065	0.00717682067576655\\
102.963672672673	0.00222051096549388\\
134.241635635636	0.000188533596550874\\
168.901	1.22612249866183e-05\\
};
\addlegendentry{$RC$ Decay}

\addplot [color=mycolor4, dashed, line width=2.0pt]
  table[row sep=crcr]{%
0	2.95093012304211\\
97.3843603603604	2.81084863512705\\
168.901	2.71222917040202\\
};
\addlegendentry{Long Decay}

\end{axis}
\end{tikzpicture}%
    \caption{Experimental data with fit to $V = A\exp(Bt) + C\exp(Dt)$ with $r^2
      = 0.9999$. \textit{$RC$ decay} shows just the first exponential
      $A\exp(Bt)$ while\textit{ Long Decay} shows the second exponential
      $C\exp(Dt)$.\label{fig:complex_fit}}
  \end{subfigure}
  \hfill
  \begin{subfigure}{0.48\textwidth}
    \centering
    % This file was created by matlab2tikz.
%
\definecolor{mycolor1}{rgb}{0.00000,0.44700,0.74100}%
%
\begin{tikzpicture}[%
trim axis left
]

\begin{axis}[%
width=1.902in,
height=2in,
at={(0in,0in)},
scale only axis,
xmin=0,
xmax=180,
xlabel style={font=\color{white!15!black}},
xlabel={Time (\si{\second})},
ymin=-0.04,
ymax=0.08,
ylabel style={font=\color{white!15!black}},
ylabel={Voltage (\si{\volt})},
axis background/.style={fill=white},
axis x line*=bottom,
axis y line*=left,
legend style={font=\small}
]
\addplot[only marks, mark=x, mark options={}, mark size=0.2500pt, draw=mycolor1] table[row sep=crcr]{%
x	y\\
0	0.0729406327292352\\
0.0999999999999996	0.0676157792112821\\
0.2	0.0628312509123354\\
0.3	0.0595906579840193\\
0.4	0.0548975822244291\\
0.5	0.0517555773008169\\
0.6	0.0491681689705334\\
0.7	0.0451388553002214\\
0.8	0.0426711068832883\\
0.899999999999999	0.0397683670556876\\
0.999999999999999	0.0374340521099921\\
1.1	0.0346715515077776\\
1.2	0.032484228090361\\
1.3	0.0298754182878742\\
1.4	0.0268484323266716\\
1.5	0.0254065544351594\\
1.6	0.022553043047953\\
1.7	0.0212911310084785\\
1.8	0.0196240257699412\\
1.9	0.0175549095947609\\
2	0.0160869397523857\\
2.1	0.0142232487156111\\
2.2	0.0119669443553132\\
2.299	0.0108294865836882\\
2.399	0.00880103138783106\\
2.499	0.00838911056032643\\
2.599	0.00659673571891872\\
2.699	0.00442689482860636\\
2.799	0.00288255238741364\\
2.899	0.00196664961069715\\
2.999	0.000682104614000423\\
3.099	-0.000968187405527843\\
3.199	-0.000981353989118361\\
3.299	-0.00235454523782863\\
3.399	-0.00408493363532969\\
3.499	-0.00416971387214105\\
3.599	-0.00460610267120387\\
3.699	-0.00639133861485597\\
3.799	-0.00752268197314443\\
3.899	-0.00799741453349512\\
3.999	-0.00881283943172129\\
4.099	-0.00896628098434959\\
4.199	-0.0104550845222811\\
4.299	-0.0112766162257376\\
4.399	-0.0114282629605249\\
4.499	-0.0119074321155637\\
4.599	-0.0127115514417007\\
4.699	-0.0138380688918076\\
4.799	-0.0132844524620932\\
4.899	-0.0150481900347064\\
4.999	-0.0151267892215463\\
5.099	-0.0155177772093094\\
5.199	-0.0162187006057506\\
5.299	-0.01622712528716\\
5.399	-0.0165406362470213\\
5.499	-0.0171568374458761\\
5.599	-0.0170733516623569\\
5.699	-0.0182878203453951\\
5.799	-0.0187979034675845\\
5.899	-0.0186012793796992\\
5.999	-0.0186956446663595\\
6.099	-0.0190787140028208\\
6.199	-0.0197482200128967\\
6.299	-0.0197019131279914\\
6.399	-0.0199375614472395\\
6.499	-0.0204529505987576\\
6.599	-0.0202458836019588\\
6.699	-0.020314180730975\\
6.799	-0.0206556793791455\\
6.899	-0.0212682339245465\\
6.999	-0.0211497155966169\\
7.099	-0.0212980123437969\\
7.199	-0.0217110287022289\\
7.299	-0.0213866856654734\\
7.399	-0.0213229205552574\\
7.499	-0.0215176868932376\\
7.599	-0.0219689542737633\\
7.699	-0.0226747082376564\\
7.799	-0.0226329501469547\\
7.899	-0.0218416970606663\\
7.999	-0.022298981611482\\
8.099	-0.0220028518834656\\
8.199	-0.0219513712907009\\
8.299	-0.0221426184568907\\
8.399	-0.021574687095903\\
8.499	-0.0222456858932576\\
8.599	-0.0221537383885391\\
8.699	-0.0222969828587383\\
8.799	-0.0216735722025021\\
8.899	-0.0222816738253009\\
8.999	-0.0221194695254967\\
9.099	-0.0221851553813019\\
9.199	-0.0214769416386185\\
9.299	-0.0219930525997851\\
9.399	-0.0217317265131678\\
9.499	-0.0216912154636297\\
9.599	-0.0218697852638705\\
9.699	-0.0212657153466003\\
9.799	-0.0218772986575795\\
9.899	-0.0217028415494793\\
9.999	-0.0217406636765967\\
10.099	-0.0209890978903786\\
10.199	-0.0214464901357738\\
10.299	-0.0211111993484039\\
10.399	-0.0209815973525318\\
10.499	-0.0210560687598376\\
10.599	-0.0213330108689931\\
10.699	-0.0208108335660109\\
10.799	-0.0204879592253926\\
10.899	-0.0203628226120447\\
10.999	-0.0204338707839673\\
11.099	-0.0196995629957035\\
11.199	-0.0201583706025508\\
11.299	-0.0198087769655277\\
11.399	-0.0196492773570753\\
11.499	-0.0196783788675132\\
11.599	-0.0188946003122226\\
11.699	-0.0192964721395574\\
11.799	-0.01888253633948\\
11.899	-0.0186513463529225\\
11.999	-0.0186014669818437\\
12.099	-0.0177314743000103\\
12.199	-0.0180399555644577\\
12.299	-0.0185255091276737\\
12.399	-0.0181867443504453\\
12.499	-0.0170222815154029\\
12.599	-0.0170307517412507\\
12.699	-0.0172107968976496\\
12.799	-0.0175610695207862\\
12.899	-0.0170802327296053\\
12.999	-0.0167669601426876\\
13.099	-0.0156199357957894\\
13.199	-0.0166378540600247\\
13.299	-0.0158194195607004\\
13.399	-0.016163347096775\\
13.499	-0.0156683615609587\\
13.599	-0.015333197860441\\
13.699	-0.0151566008382353\\
13.799	-0.0151373251951421\\
13.899	-0.0142741354123377\\
13.999	-0.0145658056745512\\
14.099	-0.0140111197938548\\
14.199	-0.0136088711340587\\
14.299	-0.0133578625356918\\
14.399	-0.0132569062415637\\
14.499	-0.0133048238229323\\
14.599	-0.0125004461062383\\
14.699	-0.012842613100406\\
14.799	-0.0123301739247408\\
14.899	-0.0119619867373668\\
14.999	-0.0117369186642406\\
15.099	-0.0116538457287207\\
15.199	-0.0117116527816874\\
15.299	-0.0109092334322138\\
15.399	-0.0102454899787787\\
15.499	-0.010719333341024\\
15.599	-0.0103296829920456\\
15.699	-0.0100754668912106\\
15.799	-0.00995562141751449\\
15.899	-0.00996909130345092\\
15.999	-0.00911482956940635\\
16.099	-0.00939179745856666\\
16.199	-0.00879896437233718\\
16.299	-0.0093353078062739\\
16.399	-0.00799981328650468\\
16.499	-0.00779147430666427\\
16.599	-0.00770929226531791\\
16.699	-0.00775227640387666\\
16.799	-0.00691944374500508\\
16.899	-0.00720981903149909\\
16.999	-0.00662243466566181\\
17.099	-0.00715633064914645\\
17.199	-0.0068105545232644\\
17.299	-0.00658416130978701\\
17.399	-0.00647621345216987\\
17.499	-0.005485780757291\\
17.599	-0.00561194033759715\\
17.699	-0.00485377655374464\\
17.799	-0.0052103809576618\\
17.899	-0.00468085223607417\\
17.999	-0.00426429615447166\\
18.099	-0.00395982550150364\\
18.199	-0.0047665600338318\\
18.299	-0.00368362642139974\\
18.399	-0.00371015819313225\\
18.499	-0.00384529568307546\\
18.599	-0.00308818597694227\\
18.699	-0.00343798285908914\\
18.799	-0.00289384675991045\\
18.899	-0.00245494470363372\\
18.999	-0.00212045025653662\\
19.099	-0.00188954347556702\\
19.199	-0.00176141085736514\\
19.299	-0.00173524528769065\\
19.399	-0.000810245991239356\\
19.499	-0.00098561848185863\\
19.599	-0.000260574513155198\\
19.699	0.000365667970508632\\
19.799	-0.000106115117357142\\
19.899	0.000324846042861537\\
19.999	0.000659315224748269\\
20.099	-0.000101949796659895\\
20.199	0.00104180280223698\\
20.299	0.00109131894034498\\
20.399	0.00104733867824969\\
20.499	0.000910596264227514\\
20.599	0.00168182017989604\\
20.699	0.00136173318549737\\
20.799	0.000951052364845495\\
20.899	0.00245048916989532\\
20.999	0.00186074946497872\\
21.099	0.00218253357069464\\
21.199	0.00241653630744132\\
21.299	0.00256344703861355\\
21.399	0.0026239497134668\\
21.499	0.00359872290963725\\
21.599	0.00348843987532454\\
21.699	0.00329376857115626\\
21.799	0.00401537171170752\\
21.899	0.00465390680670819\\
21.999	0.00421002620192201\\
22.098	0.00458003625182268\\
22.198	0.0049740689327189\\
22.298	0.00528760602515099\\
22.398	0.00452127967124127\\
22.498	0.00567571704837899\\
22.598	0.00575154040820713\\
22.698	0.00574936711531215\\
22.798	0.00566980968560404\\
22.898	0.00551347582439909\\
22.998	0.0062809684641989\\
23.098	0.00597288580217636\\
23.198	0.00658982133737052\\
23.298	0.00713236390757288\\
23.398	0.00660109772595252\\
23.498	0.00699660241736222\\
23.598	0.00831945305438531\\
23.698	0.00757022019308007\\
23.798	0.00774946990845748\\
23.898	0.00785776382967196\\
23.998	0.00789565917493906\\
24.098	0.00786370878618037\\
24.198	0.0087624611633883\\
24.298	0.00859246049873263\\
24.398	0.00935424671039176\\
24.498	0.00904835547612004\\
24.598	0.00867531826655021\\
24.698	0.00923566237823481\\
24.798	0.00972991096643172\\
24.898	0.00915858307762774\\
24.998	0.00952219368180707\\
25.098	0.00982125370446774\\
25.198	0.0100562700583895\\
25.298	0.010227745675143\\
25.398	0.0103361795363646\\
25.498	0.0103820667047736\\
25.598	0.0113658983549501\\
25.698	0.010288161803877\\
25.798	0.0111493405412313\\
25.898	0.0109499142594469\\
25.998	0.010690358883537\\
26.098	0.0113711466006783\\
26.198	0.0119927458895757\\
26.298	0.0115556215495798\\
26.398	0.0120602347295899\\
26.498	0.0115070429567194\\
26.598	0.011896500164744\\
26.698	0.0112290567223234\\
26.798	0.0125051594609973\\
26.898	0.0127252517029692\\
26.998	0.0128897732886668\\
27.098	0.0129991606040858\\
27.198	0.0130538466079257\\
27.298	0.0130542608584987\\
27.398	0.0130008295404433\\
27.498	0.0128939754912176\\
27.598	0.0137341182273869\\
27.698	0.0135216739707067\\
27.798	0.0142570556739985\\
27.898	0.0139406730468261\\
27.998	0.0135729325809613\\
28.098	0.0141542375756645\\
28.198	0.0136849881627521\\
28.298	0.0141655813314743\\
28.398	0.0145964109531973\\
28.498	0.0139778678058899\\
28.598	0.0143103395984201\\
28.698	0.0135942109946554\\
28.798	0.0148298636373823\\
28.898	0.0150176761720298\\
28.998	0.0151580242702143\\
29.098	0.0152512806530898\\
29.198	0.0142978151145252\\
29.298	0.0152979945440914\\
29.398	0.0152521829498755\\
29.498	0.0151607414811048\\
29.598	0.0160240284506061\\
29.698	0.0148423993570796\\
29.798	0.0156162069071999\\
29.898	0.0153458010375469\\
29.998	0.0160315289363568\\
30.098	0.0156737350651133\\
30.198	0.0162727611799576\\
30.298	0.0158289463529391\\
30.398	0.0163426269930946\\
30.498	0.0158141368673634\\
30.598	0.0162438071213367\\
30.698	0.016631966299848\\
30.798	0.0159789403673956\\
30.898	0.0162850527284113\\
30.998	0.0165506242473645\\
31.098	0.0167759732687109\\
31.198	0.0169614156366853\\
31.298	0.0171072647149377\\
31.398	0.017213831406016\\
31.498	0.0172814241706916\\
31.598	0.0173103490471407\\
31.698	0.0173009096699701\\
31.798	0.0172534072890929\\
31.898	0.0171681407884594\\
31.998	0.0170454067046375\\
32.098	0.0178854992452475\\
32.198	0.0176887103072585\\
32.298	0.0174553294951298\\
32.398	0.0171856441388192\\
32.498	0.0178799393116451\\
32.598	0.0175384978480113\\
32.698	0.01716160036099\\
32.798	0.0177495252597675\\
32.898	0.0183025487669539\\
32.998	0.0178209449357549\\
33.098	0.0173049856670131\\
33.198	0.0177549407261091\\
33.298	0.0181710777597321\\
33.398	0.0175536623125243\\
33.498	0.0179029578435856\\
33.598	0.0182192257428557\\
33.698	0.0175027253473656\\
33.798	0.0177537139573558\\
33.898	0.0179724468522786\\
33.998	0.0181591773066674\\
34.098	0.0183141566058822\\
34.198	0.0184376340617312\\
34.298	0.0185298570279731\\
34.398	0.0185910709156945\\
34.498	0.0186215192085659\\
34.598	0.0186214434779832\\
34.698	0.0185910833980798\\
34.798	0.0185306767606339\\
34.898	0.0184404594898493\\
34.998	0.0183206656570221\\
35.098	0.0181715274950958\\
35.198	0.0179932754130969\\
35.298	0.0187861380104604\\
35.398	0.0185503420912441\\
35.498	0.0182861126782243\\
35.598	0.018993673026892\\
35.698	0.0186732446393285\\
35.798	0.0183250472779757\\
35.898	0.0189492989793036\\
35.998	0.0185462160673606\\
36.098	0.0181160131672273\\
36.198	0.0186589032183551\\
36.298	0.0191750974878087\\
36.398	0.0186648055833989\\
36.498	0.0191282354667144\\
36.598	0.018565593466052\\
36.698	0.0189770842892427\\
36.798	0.0183629110363781\\
36.898	0.0187232752124404\\
36.998	0.0190583767398258\\
37.098	0.019368413970779\\
37.198	0.0186535836997193\\
37.298	0.0189140811754802\\
37.398	0.0191501001134453\\
37.498	0.0183618327075923\\
37.598	0.0185494696424437\\
37.698	0.0187132001049193\\
37.798	0.0188532117960993\\
37.898	0.0189696909428942\\
37.998	0.0190628223096212\\
38.098	0.0191327892094941\\
38.198	0.019179773516016\\
38.298	0.01920395567429\\
38.398	0.0192055147122354\\
38.498	0.0191846282517205\\
38.598	0.0191414725196015\\
38.698	0.0180762223586837\\
38.798	0.0189890512385857\\
38.898	0.0188801312665308\\
38.998	0.0187496331980426\\
39.098	0.0185977264475605\\
39.198	0.0184245790989754\\
39.298	0.0182303579160799\\
39.398	0.0190152283529299\\
39.498	0.0187793545641397\\
39.598	0.0185228994150823\\
39.698	0.0182460244920142\\
39.798	0.0189488901121253\\
39.898	0.0186316553335044\\
39.998	0.0182944779650249\\
40.098	0.0179375145761593\\
40.198	0.0185609205067152\\
40.298	0.0181648498764853\\
40.398	0.0187494555948375\\
40.498	0.0183148893702167\\
40.598	0.0178613017195786\\
40.698	0.0183888419777438\\
40.798	0.0178976583066897\\
40.898	0.018387897704752\\
40.998	0.0178597060157721\\
41.098	0.0183132279381617\\
41.198	0.0187486070338938\\
41.298	0.0181659857374368\\
41.398	0.0185655053646028\\
41.498	0.017947306121338\\
41.598	0.0183115271124352\\
41.698	0.0186583063501846\\
41.797	0.0179645712307512\\
41.898	0.0183000862037055\\
41.997	0.0175724886251345\\
42.097	0.0178510277673332\\
42.197	0.0181127978064559\\
42.297	0.017357930388004\\
42.397	0.0175865561235553\\
42.497	0.0177988045988919\\
42.597	0.0179948043820528\\
42.697	0.0171746830313273\\
42.797	0.0173385671031867\\
42.897	0.0174865821601515\\
42.997	0.0166188527785982\\
43.097	0.0177355025565031\\
43.197	0.0168366541211284\\
43.297	0.0169224291366441\\
43.397	0.0169929483116942\\
43.497	0.0170483314068983\\
43.597	0.017088697242301\\
43.697	0.0171141637047545\\
43.797	0.0171248477552521\\
43.897	0.0171208654361967\\
43.997	0.017102331878617\\
44.097	0.0160693613093232\\
44.197	0.0170220670580128\\
44.297	0.0169605615643098\\
44.397	0.0158849563847618\\
44.497	0.0167953621997725\\
44.597	0.0166918888204814\\
44.697	0.0165746451955937\\
44.797	0.0164437394181522\\
44.897	0.0162992787322578\\
44.997	0.0161413695397363\\
45.097	0.0159701174067548\\
45.197	0.0167856270703841\\
45.297	0.0165880024451099\\
45.397	0.016377346629294\\
45.497	0.0161537619115841\\
45.597	0.0159173497772715\\
45.697	0.0156682109146034\\
45.797	0.0154064452210387\\
45.897	0.0161321518094644\\
45.997	0.0158454290143508\\
46.097	0.0155463743978719\\
46.197	0.016235084755964\\
46.297	0.0159116561243495\\
46.397	0.0155761837845043\\
46.497	0.0152287622695817\\
46.597	0.0148694853702924\\
46.697	0.0154984461407293\\
46.797	0.0151157369041606\\
46.897	0.0147214492587642\\
46.997	0.0153156740833249\\
47.097	0.0148985015428811\\
47.197	0.0144700210943349\\
47.297	0.0150303214920116\\
47.397	0.0145794907931762\\
47.497	0.0151176163635118\\
47.597	0.0146447848825466\\
47.697	0.0141610823490459\\
47.797	0.014666594086358\\
47.897	0.0141614047477177\\
47.997	0.0146455983215117\\
48.097	0.0141192581364984\\
48.197	0.0135824668669868\\
48.297	0.0140353065379815\\
48.397	0.0144778585302774\\
48.497	0.0129102035855211\\
48.597	0.0133324218112314\\
48.697	0.0127445926857774\\
48.797	0.0141467950633229\\
48.897	0.0135391071787261\\
48.997	0.0129216066524038\\
49.097	0.0132943704951587\\
49.197	0.0136574751129674\\
49.297	0.013010996311726\\
49.397	0.013355009301971\\
49.497	0.0126895887035463\\
49.597	0.0130148085502513\\
49.697	0.0133307422944351\\
49.797	0.0126374628115689\\
49.897	0.0129350424047754\\
49.997	0.0122235528093229\\
50.097	0.0125030651970834\\
50.197	0.0127736501809648\\
50.297	0.0130353778192913\\
50.397	0.0122883176201651\\
50.497	0.0115325385457865\\
50.597	0.01276810901674\\
50.697	0.0119950969162494\\
50.797	0.0122135695943992\\
50.897	0.0114235938723182\\
50.997	0.0116252360463367\\
51.097	0.0118185618921092\\
51.197	0.0120036366686986\\
51.297	0.01118052512264\\
51.397	0.0113492914919608\\
51.497	0.0115099995101775\\
51.597	0.0116627124102568\\
51.697	0.0118074929285465\\
51.797	0.0119444033086769\\
51.897	0.0110735053054305\\
51.997	0.0111948601885787\\
52.097	0.0113085287466963\\
52.197	0.0104145712909349\\
52.297	0.010513047658776\\
52.397	0.0106040172177511\\
52.497	0.010687538869131\\
52.597	0.0107636710515888\\
52.697	0.00983247174483148\\
52.797	0.00989399847320893\\
52.897	0.00994830830928439\\
52.997	0.00999545787738665\\
53.097	0.0100355033571287\\
53.197	0.0100685004869017\\
53.297	0.00909450456733962\\
53.397	0.0101135704647564\\
53.497	0.00912575261455872\\
53.597	0.00913110502462855\\
53.697	0.00912968127868297\\
53.797	0.0101215345396031\\
53.897	0.00910671755274128\\
53.997	0.00908528264919983\\
54.097	0.0090572817490826\\
54.197	0.00902276636472532\\
54.297	0.00898178760389712\\
54.397	0.00893439617297886\\
54.497	0.00888064238011266\\
54.597	0.00882057613833309\\
54.697	0.00875424696866967\\
54.797	0.00868170400322388\\
54.897	0.00860299598822367\\
54.997	0.00851817128705612\\
55.097	0.00842727788327347\\
55.197	0.00833036338357562\\
55.297	0.00822747502076915\\
55.397	0.00811865965670622\\
55.497	0.00800396378519697\\
55.597	0.00788343353489784\\
55.697	0.00775711467218221\\
55.797	0.00762505260398472\\
55.897	0.00748729238062573\\
55.997	0.00734387869860864\\
56.097	0.00719485590340074\\
56.197	0.00704026799219193\\
56.297	0.0078801586166275\\
56.397	0.00671457108552254\\
56.497	0.00754354836755411\\
56.597	0.00636713309393411\\
56.697	0.00718536756106003\\
56.797	0.00599829373314087\\
56.897	0.0068059532448097\\
56.997	0.00660838740370995\\
57.097	0.00640563719306675\\
57.197	0.00619774327423128\\
57.297	0.00698474598920829\\
57.397	0.00676668536316916\\
57.497	0.00654360110693508\\
57.597	0.00631553261944751\\
57.697	0.00608251899021983\\
57.797	0.00584459900176393\\
57.897	0.0056018111320042\\
57.997	0.00635419355666755\\
58.097	0.00510178415165896\\
58.197	0.00584462049541568\\
58.297	0.00458273987123992\\
58.397	0.00531617926962369\\
58.497	0.00504497539054194\\
58.597	0.00476916464573796\\
58.697	0.00548878316098644\\
58.797	0.00520386677833828\\
58.897	0.00491445105835098\\
58.997	0.00462057128229443\\
59.097	0.00532226245435297\\
59.197	0.00501955930379516\\
59.297	0.00471249628713188\\
59.397	0.00440110759026391\\
59.497	0.00408542713060367\\
59.597	0.00476548855918457\\
59.697	0.00444132526275309\\
59.797	0.00411297036584513\\
59.897	0.00378045673284388\\
59.997	0.00344381697002394\\
60.097	0.00410308342758015\\
60.197	0.00375828820163537\\
60.297	0.00340946313623736\\
60.397	0.00405663982534055\\
60.497	0.00369984961476666\\
60.597	0.00333912360415756\\
60.697	0.00397449264890426\\
60.797	0.00360598736206841\\
60.897	0.00323363811628541\\
60.997	0.00285747504564959\\
61.097	0.00347752804759205\\
61.197	0.00309382678473602\\
61.297	0.00270640068674366\\
61.397	0.00231527895214212\\
61.497	0.00192049055014287\\
61.597	0.00252206422243972\\
61.697	0.00212002848499715\\
61.797	0.00271441162982233\\
61.896	0.00229935092171862\\
61.996	0.00188669093490823\\
62.096	0.00247053310493772\\
62.196	0.00205090484760184\\
62.296	0.00262783336337691\\
62.396	0.002201345639115\\
62.496	0.00177146844972098\\
62.596	0.00233822835981634\\
62.696	0.00190165172539114\\
62.796	0.00146176469544246\\
62.896	0.00201859321360098\\
62.996	0.0015721630197425\\
63.096	0.00112249965158995\\
63.196	0.000669628446299253\\
63.296	0.00121357454203563\\
63.396	0.000754362879535453\\
63.496	0.00129201820365843\\
63.596	0.000826565064923734\\
63.696	0.000358027821040441\\
63.796	0.000886430638415536\\
63.896	0.000411797493661048\\
63.996	0.000934152175084968\\
64.096	0.00145351828416729\\
64.196	0.000969919237028272\\
64.296	0.000483378265886181\\
64.396	0.000993918420498119\\
64.496	0.000501562569596992\\
64.596	6.33340231104285e-06\\
64.696	0.000508253429572392\\
64.796	7.34498552157348e-06\\
64.896	-0.000496369771108895\\
64.996	-2.86885561262551e-06\\
65.096	-0.000512130455945226\\
65.196	-2.41329313692695e-05\\
65.296	-0.000538854811106049\\
65.396	-5.62747930015206e-05\\
65.496	-0.000576371742203374\\
65.596	-0.00109912468984597\\
65.696	-0.000624512831746582\\
65.796	-0.00115251552711193\\
65.896	-0.00168311229725582\\
65.996	-0.00121628282432606\\
66.096	-0.000752006950039963\\
66.196	-0.00129026467443083\\
66.296	-0.000831036154608\\
66.396	-0.00137430170351927\\
66.496	-0.00192004178872818\\
66.596	-0.00146823703120091\\
66.696	-0.00101886820409902\\
66.796	-0.00157191623158948\\
66.896	-0.00212736218764942\\
66.996	-0.00168518729489975\\
67.096	-0.00224537292343019\\
67.196	-0.00180790058964497\\
67.296	-0.00237275195511266\\
67.396	-0.00193990882542572\\
67.496	-0.00250935314906942\\
67.596	-0.0020810670163014\\
67.696	-0.00265503265803435\\
67.796	-0.00223123244473511\\
67.896	-0.00280964888532731\\
67.996	-0.00239026462610203\\
68.096	-0.00297306244964357\\
68.196	-0.00255802527375337\\
68.296	-0.00314513615039225\\
68.396	-0.00273437826462519\\
68.496	-0.00232573493357702\\
68.596	-0.00291918960539217\\
68.696	-0.00351472585820956\\
68.796	-0.00311232739913914\\
68.896	-0.00371197806324997\\
68.996	-0.00331366181256465\\
69.096	-0.00391736273506282\\
69.196	-0.0035230650436926\\
69.296	-0.00413075307538646\\
69.396	-0.00374041129009139\\
69.496	-0.0033520242697973\\
69.596	-0.00396557671758613\\
69.696	-0.00358105345667292\\
69.796	-0.00319843942946729\\
69.896	-0.00381771969663625\\
69.996	-0.0044388794361736\\
70.096	-0.00406190394248229\\
70.196	-0.00468677862545519\\
70.296	-0.00431348900957351\\
70.396	-0.00494202073300087\\
70.496	-0.00457235954669599\\
70.596	-0.00520449131352363\\
70.696	-0.00483840200737573\\
70.796	-0.00447407771230113\\
70.896	-0.00511150462163901\\
70.996	-0.00475066903716259\\
71.096	-0.00539155736822261\\
71.196	-0.00503415613090796\\
71.296	-0.00467845194720651\\
71.396	-0.00432443154416839\\
71.496	-0.00497208175308694\\
71.596	-0.00462138950867796\\
71.696	-0.00527234184826542\\
71.796	-0.00492492591097715\\
71.896	-0.00557912893694779\\
71.996	-0.00523493826652022\\
72.096	-0.0058923413394627\\
72.196	-0.0055513256941877\\
72.296	-0.00521187896697395\\
72.396	-0.00587398889120072\\
72.496	-0.00553764329658657\\
72.596	-0.00620283010842737\\
72.696	-0.00586953734685158\\
72.796	-0.00553775312607163\\
72.896	-0.00620746565364483\\
72.996	-0.00687866322974484\\
73.096	-0.00655133424642873\\
73.196	-0.00722546718691852\\
73.296	-0.006901050624883\\
73.396	-0.00657807322373127\\
73.496	-0.00725652373590258\\
73.596	-0.00693639100217114\\
73.696	-0.00661766395095187\\
73.796	-0.00730033159760923\\
73.896	-0.00698438304377857\\
73.996	-0.00766980747668544\\
74.096	-0.00735659416847412\\
74.196	-0.0070447324755416\\
74.296	-0.00673421183787459\\
74.396	-0.00742502177839333\\
74.496	-0.00711715190230189\\
74.596	-0.00681059189643918\\
74.696	-0.00750533152864064\\
74.796	-0.00720136064709864\\
74.896	-0.00789866917973425\\
74.996	-0.00759724713357102\\
75.096	-0.00729708459411293\\
75.196	-0.00799817172472617\\
75.296	-0.00770049876603185\\
75.396	-0.00740405603529526\\
75.496	-0.00810883392582529\\
75.596	-0.0078148229063788\\
75.696	-0.00752201352056625\\
75.796	-0.00823039638626355\\
75.896	-0.0079399621950289\\
75.996	-0.00765070171152704\\
76.096	-0.00836260577295089\\
76.196	-0.00907566528845383\\
76.296	-0.00878987123858499\\
76.396	-0.00850521467472865\\
76.496	-0.00922168671854529\\
76.596	-0.00893927856142396\\
76.696	-0.00865798146393093\\
76.796	-0.00837778675526746\\
76.896	-0.00909868583273221\\
76.996	-0.00882067016118437\\
77.096	-0.00854373127251451\\
77.196	-0.00926786076511599\\
77.296	-0.00999305030336739\\
77.396	-0.00971929161710783\\
77.496	-0.00944657650112823\\
77.596	-0.00917489681465877\\
77.696	-0.00890424448086202\\
77.796	-0.00863461148633249\\
77.896	-0.00936598988059911\\
77.996	-0.00909837177562789\\
78.096	-0.00983174934533482\\
78.196	-0.00956611482509606\\
78.296	-0.0093014605112689\\
78.396	-0.0100377787607102\\
78.496	-0.00977506199030298\\
78.596	-0.0095133026764822\\
78.696	-0.00925249335476996\\
78.796	-0.00899262661931122\\
78.896	-0.00973369512241051\\
78.996	-0.0104756915740789\\
79.096	-0.010218608741579\\
79.196	-0.0099624394489739\\
79.296	-0.00970717657668363\\
79.396	-0.00945281306104162\\
79.496	-0.0101993418938555\\
79.596	-0.00994675612197016\\
79.696	-0.00969504884683747\\
79.796	-0.00944421322408617\\
79.896	-0.0101942424630983\\
79.996	-0.00994512982658335\\
80.096	-0.0106968686301636\\
80.196	-0.0104494522419545\\
80.296	-0.0112028740821559\\
80.396	-0.00995712762264178\\
80.496	-0.0107122063865526\\
80.596	-0.0104681039478964\\
80.696	-0.0112248139311455\\
80.796	-0.0109823300108416\\
80.896	-0.0107406459112038\\
80.996	-0.0104997554057373\\
81.096	-0.0112596523168458\\
81.196	-0.0110203305154482\\
81.296	-0.0107817839205979\\
81.395	-0.0115463804856044\\
81.496	-0.0113069922651561\\
81.595	-0.0110730941208095\\
81.695	-0.010837581007638\\
81.796	-0.0106004695334092\\
81.896	-0.0113664491262013\\
81.996	-0.0111331626752809\\
82.095	-0.0109029264672618\\
82.196	-0.0116687688501882\\
82.295	-0.0114399578476627\\
82.395	-0.0122095434897109\\
82.495	-0.0109798350418218\\
82.595	-0.0117508270150148\\
82.695	-0.0115225139634156\\
82.795	-0.0112948904839176\\
82.895	-0.0120679512158479\\
82.995	-0.0118416908406322\\
83.095	-0.011616104081464\\
83.195	-0.0123911857029779\\
83.295	-0.0121669305109227\\
83.395	-0.0119433333518404\\
83.495	-0.0117203891127433\\
83.595	-0.0124980927207994\\
83.695	-0.0122764391430121\\
83.795	-0.012055423385914\\
83.895	-0.0128350404952493\\
83.995	-0.0126152855556723\\
84.095	-0.0123961536904367\\
84.195	-0.0121776400610965\\
84.295	-0.0129597398672008\\
84.395	-0.0127424483460006\\
84.495	-0.0125257607721481\\
84.595	-0.0123096724574063\\
84.695	-0.0120941787503543\\
84.795	-0.0128792750361013\\
84.895	-0.0126649567359984\\
84.995	-0.0124512193073523\\
85.095	-0.0132380582431475\\
85.195	-0.0130254690717595\\
85.295	-0.0128134473566823\\
85.395	-0.0126019886962516\\
85.495	-0.0123910887233687\\
85.595	-0.0131807431052313\\
85.695	-0.0129709475430633\\
85.795	-0.0127616977718485\\
85.895	-0.0135529895600626\\
85.995	-0.013344818709415\\
86.095	-0.0131371810545828\\
86.195	-0.0129300724629551\\
86.295	-0.0127234888343755\\
86.395	-0.0125174261008856\\
86.495	-0.0133118802264738\\
86.595	-0.0131068472068243\\
86.695	-0.0129023230690679\\
86.795	-0.0136983038715344\\
86.895	-0.0124947857035074\\
86.995	-0.0132917646849839\\
87.095	-0.0130892369664299\\
87.195	-0.0128871987285413\\
87.295	-0.0136856461820081\\
87.395	-0.0134845755672792\\
87.495	-0.0132839831543272\\
87.595	-0.013083865242415\\
87.695	-0.0128842181598712\\
87.795	-0.012685038263855\\
87.895	-0.0134863219401358\\
87.995	-0.0132880656028633\\
88.095	-0.0140902656943491\\
88.195	-0.0128929186848437\\
88.295	-0.0136960210723145\\
88.395	-0.0134995693822333\\
88.495	-0.014303560167356\\
88.595	-0.0131079900075095\\
88.695	-0.0139128555093819\\
88.795	-0.0137181533063058\\
88.895	-0.0145238800580536\\
88.995	-0.0133300324506296\\
89.095	-0.0131366071960617\\
89.195	-0.0139436010321989\\
89.295	-0.013751010722507\\
89.395	-0.0135588330558689\\
89.495	-0.0143670648463847\\
89.595	-0.0141757029331719\\
89.695	-0.0139847441801724\\
89.795	-0.0137941854759536\\
89.895	-0.0136040237335173\\
89.995	-0.0144142558901077\\
90.095	-0.0142248789070205\\
90.195	-0.0140358897694139\\
90.295	-0.0148472854861224\\
90.395	-0.0136590630894675\\
90.495	-0.014471219635078\\
90.595	-0.0142837522017043\\
90.695	-0.0140966578910362\\
90.795	-0.0139099338275228\\
90.895	-0.0147235771581968\\
90.995	-0.0145375850524938\\
91.095	-0.014351954702077\\
91.195	-0.0141666833206631\\
91.295	-0.0139817681438505\\
91.395	-0.0137972064289436\\
91.495	-0.0146129954547871\\
91.595	-0.0144291325215926\\
91.695	-0.0142456149507746\\
91.795	-0.0140624400847793\\
91.895	-0.0138796052869248\\
91.995	-0.0146971079412337\\
92.095	-0.0145149454522704\\
92.195	-0.0143331152449817\\
92.295	-0.0141516147645349\\
92.395	-0.0139704414761606\\
92.495	-0.013789592864994\\
92.595	-0.0136090664359187\\
92.695	-0.0144288597134121\\
92.795	-0.0142489702413902\\
92.895	-0.014069395583058\\
92.995	-0.0148901333207538\\
93.095	-0.0147111810558025\\
93.195	-0.0145325364083644\\
93.295	-0.0143541970172878\\
93.395	-0.0141761605399631\\
93.495	-0.014998424652175\\
93.595	-0.0148209870479619\\
93.695	-0.0146438454394673\\
93.795	-0.0144669975568026\\
93.895	-0.0142904411479021\\
93.995	-0.0141141739783861\\
94.095	-0.0149381938314201\\
94.195	-0.0147624985075763\\
94.295	-0.0145870858247008\\
94.395	-0.0144119536177731\\
94.495	-0.0152370997387741\\
94.595	-0.0150625220565517\\
94.695	-0.0148882184566905\\
94.795	-0.0147141868413767\\
94.895	-0.0145404251292707\\
94.995	-0.0143669312553762\\
95.095	-0.0151937031709131\\
95.195	-0.0150207388431882\\
95.295	-0.0148480362554708\\
95.395	-0.0146755934068659\\
95.495	-0.0145034083121915\\
95.595	-0.0153314790018531\\
95.695	-0.0151598035217222\\
95.795	-0.0149883799330159\\
95.895	-0.0148172063121756\\
95.995	-0.0146462807507461\\
96.095	-0.0154756013552584\\
96.195	-0.0153051662471135\\
96.295	-0.0151349735624606\\
96.395	-0.0149650214520878\\
96.495	-0.0147953080813021\\
96.595	-0.0146258316298171\\
96.695	-0.0144565902916414\\
96.795	-0.0152875822749627\\
96.895	-0.0151188058020404\\
96.995	-0.0149502591090913\\
97.095	-0.0147819404461838\\
97.195	-0.0156138480771264\\
97.295	-0.0154459802793609\\
97.395	-0.0152783353438548\\
97.495	-0.0151109115749959\\
97.595	-0.014943707290485\\
97.695	-0.0157767208212336\\
97.795	-0.0156099505112599\\
97.895	-0.0154433947175834\\
97.995	-0.0152770518101262\\
98.095	-0.0151109201716091\\
98.195	-0.0149449981974525\\
98.295	-0.0147792842956758\\
98.395	-0.0146137768867991\\
98.495	-0.015448474403744\\
98.595	-0.0152833752917383\\
98.695	-0.0151184780082172\\
98.795	-0.014953781022728\\
98.895	-0.0147892828168357\\
98.995	-0.0146249818840269\\
99.095	-0.0144608767296202\\
99.195	-0.0152969658706672\\
99.295	-0.0151332478358652\\
99.395	-0.014969721165464\\
99.495	-0.014806384411175\\
99.595	-0.0146432361360826\\
99.695	-0.0154802749145513\\
99.795	-0.0153174993321428\\
99.895	-0.0151549079855222\\
99.995	-0.0149924994823749\\
100.095	-0.0148302724413174\\
100.195	-0.014668225491814\\
100.295	-0.0145063572740898\\
100.395	-0.0153446664390446\\
100.495	-0.0141831516481754\\
100.595	-0.0150218115734848\\
100.695	-0.0148606448974045\\
100.795	-0.0146996503127101\\
100.895	-0.0145388265224429\\
100.995	-0.0153781722398265\\
101.095	-0.0142176861881889\\
101.195	-0.01505736710088\\
101.295	-0.0148972137211967\\
101.395	-0.0147372248023041\\
101.495	-0.0145773991071541\\
101.594	-0.014419331247526\\
101.694	-0.0152598267257082\\
101.794	-0.0151004817864999\\
101.894	-0.0149412952311581\\
101.994	-0.0147822658703491\\
102.094	-0.014623392524078\\
102.194	-0.0144646740216148\\
102.294	-0.0153061092014211\\
102.394	-0.0151476969110793\\
102.494	-0.0149894360072187\\
102.594	-0.0148313253554475\\
102.694	-0.0146733638302772\\
102.794	-0.0145155503150596\\
102.894	-0.014357883701912\\
102.994	-0.0142003628916498\\
103.094	-0.0150429867937203\\
103.194	-0.01488575432613\\
103.294	-0.0147286644153839\\
103.394	-0.0145717159964125\\
103.494	-0.0144149080125104\\
103.594	-0.0142582394152679\\
103.694	-0.0151017091645067\\
103.794	-0.0149453162282147\\
103.894	-0.0147890595824829\\
103.994	-0.0146329382114407\\
104.094	-0.014476951107194\\
104.194	-0.0143210972697618\\
104.294	-0.0141653757070124\\
104.394	-0.014009785434606\\
104.494	-0.0148543254759286\\
104.594	-0.0146989948620355\\
104.694	-0.0145437926315868\\
104.794	-0.0153887178307932\\
104.894	-0.0142337695133516\\
104.994	-0.0140789467403879\\
105.094	-0.0149242485804009\\
105.194	-0.0137696741092008\\
105.294	-0.0146152224098537\\
105.394	-0.0144608925726257\\
105.494	-0.0143066836949242\\
105.594	-0.0151525948812421\\
105.694	-0.0149986252431042\\
105.794	-0.0138447738990091\\
105.894	-0.0136910399743768\\
105.994	-0.0145374226014932\\
106.094	-0.014383920919455\\
106.194	-0.01423053407412\\
106.294	-0.0140772612180498\\
106.394	-0.0149241015104598\\
106.494	-0.0137710541171656\\
106.594	-0.0136181182105313\\
106.694	-0.0144652929694193\\
106.794	-0.0143125775791382\\
106.894	-0.0141599712313907\\
106.994	-0.0140074731242277\\
107.094	-0.0138550824619945\\
107.194	-0.0137027984552813\\
107.294	-0.0135506203208773\\
107.394	-0.0133985472817195\\
107.494	-0.0142465785668451\\
107.594	-0.0140947134113447\\
107.694	-0.0139429510563107\\
107.794	-0.0137912907487969\\
107.894	-0.0136397317417649\\
107.994	-0.0134882732940418\\
108.094	-0.0133369146702735\\
108.194	-0.013185655140878\\
108.294	-0.0140344939820003\\
108.394	-0.0138834304754689\\
108.494	-0.0137324639087488\\
108.594	-0.0135815935748989\\
108.694	-0.0134308187725276\\
108.794	-0.0132801388057482\\
108.894	-0.0141295529841381\\
108.994	-0.0129790606226945\\
109.094	-0.0138286610417904\\
109.194	-0.0136783535671343\\
109.294	-0.0135281375297289\\
109.394	-0.0133780122658274\\
109.494	-0.0132279771168924\\
109.594	-0.0130780314295573\\
109.694	-0.0139281745555837\\
109.794	-0.0127784058518214\\
109.894	-0.013628724680169\\
109.994	-0.0134791304075339\\
110.094	-0.0133296224057924\\
110.194	-0.0131802000517522\\
110.294	-0.0130308627271121\\
110.394	-0.0128816098184248\\
110.494	-0.0127324407170573\\
110.594	-0.0135833548191551\\
110.694	-0.0134343515256017\\
110.794	-0.0132854302419849\\
110.894	-0.0131365903785592\\
110.994	-0.012987831350205\\
111.094	-0.0128391525763982\\
111.194	-0.0126905534811699\\
111.294	-0.0125420334930721\\
111.394	-0.0133935920451433\\
111.494	-0.0132452285748705\\
111.594	-0.0130969425241574\\
111.694	-0.0129487333392864\\
111.794	-0.0128006004708885\\
111.894	-0.0136525433739036\\
111.994	-0.0125045615075519\\
112.094	-0.0123566543352984\\
112.194	-0.0132088213248163\\
112.294	-0.013061061947961\\
112.394	-0.0129133756807298\\
112.494	-0.0127657620032338\\
112.594	-0.0126182203996659\\
112.694	-0.0134707503582643\\
112.794	-0.0123233513712875\\
112.894	-0.0131760229349753\\
112.994	-0.012028764549525\\
113.094	-0.0128815757190521\\
113.194	-0.0117344559515691\\
113.294	-0.0125874047589454\\
113.394	-0.0124404216568834\\
113.494	-0.0132935061648873\\
113.594	-0.0121466578062295\\
113.694	-0.0119998761079265\\
113.794	-0.0118531606007051\\
113.894	-0.012706510818977\\
113.994	-0.0125599263008049\\
114.094	-0.0124134065878794\\
114.194	-0.0122669512254863\\
114.294	-0.0121205597624803\\
114.394	-0.0119742317512586\\
114.494	-0.0118279667477266\\
114.594	-0.0126817643112798\\
114.694	-0.012535624004768\\
114.794	-0.0123895453944729\\
114.894	-0.0122435280500799\\
114.994	-0.0120975715446514\\
115.094	-0.0119516754545996\\
115.194	-0.0118058393596612\\
115.294	-0.01166006284287\\
115.394	-0.011514345490534\\
115.494	-0.0123686868922053\\
115.594	-0.0112230866406575\\
115.694	-0.0120775443318606\\
115.794	-0.0119320595649546\\
115.894	-0.0117866319422242\\
115.994	-0.011641261069077\\
116.094	-0.0114959465540152\\
116.194	-0.0113506880086134\\
116.294	-0.0112054850474941\\
116.394	-0.0110603372883049\\
116.494	-0.0119152443516923\\
116.594	-0.0117702058612785\\
116.694	-0.0116252214436434\\
116.794	-0.0114802907282909\\
116.894	-0.0113354133476373\\
116.994	-0.0111905889369801\\
117.094	-0.0110458171344785\\
117.194	-0.010901097581133\\
117.294	-0.0107564299207583\\
117.394	-0.010611813799966\\
117.494	-0.0114672488681387\\
117.594	-0.01032273477741\\
117.694	-0.0111782711826454\\
117.794	-0.0110338577414155\\
117.894	-0.0108894941139783\\
117.994	-0.010745179963259\\
118.094	-0.0106009149548245\\
118.194	-0.0104566987568679\\
118.294	-0.0103125310401846\\
118.394	-0.0101684114781539\\
118.494	-0.0100243397467157\\
118.594	-0.010880315524354\\
118.694	-0.0107363384920749\\
118.794	-0.00959240833338582\\
118.894	-0.0104485247342789\\
118.994	-0.0103046873832087\\
119.094	-0.0101608959710742\\
119.194	-0.0100171501911999\\
119.294	-0.00987344973931492\\
119.394	-0.0107297943135363\\
119.494	-0.00958618361434871\\
119.594	-0.0094426173445874\\
119.694	-0.0102990952094175\\
119.794	-0.0101556169163186\\
119.894	-0.0100121821750641\\
119.994	-0.00986879069770286\\
120.094	-0.00972544219854532\\
120.194	-0.00958213639414085\\
120.294	-0.0104388730032623\\
120.394	-0.00929565174688962\\
120.494	-0.00915247234819194\\
120.594	-0.0100093345325072\\
120.694	-0.00986623802733089\\
120.794	-0.00972318256229343\\
120.894	-0.00958016786914806\\
120.994	-0.00943719368175078\\
121.094	-0.00929425973604525\\
121.194	-0.00915136577004727\\
121.294	-0.00900851152382609\\
121.394	-0.00886569673949067\\
121.493	-0.00972434872372219\\
121.593	-0.0095816117092884\\
121.693	-0.00943891339764846\\
121.793	-0.00929625353888808\\
121.893	-0.00915363188505181\\
121.993	-0.00901104819012977\\
122.093	-0.00886850221004076\\
122.193	-0.00872599370261984\\
122.293	-0.00858352242759874\\
122.393	-0.00844108814659528\\
122.493	-0.00929869062309585\\
122.593	-0.00915632962244217\\
122.693	-0.00901400491181592\\
122.793	-0.00887171626022498\\
122.893	-0.00872946343848779\\
122.993	-0.0085872462192218\\
123.093	-0.00844506437682524\\
123.193	-0.00830291768746827\\
123.293	-0.00816080592907387\\
123.393	-0.00801872888130717\\
123.493	-0.0088766863255616\\
123.593	-0.00873467804494421\\
123.693	-0.00859270382426347\\
123.793	-0.00845076345001505\\
123.893	-0.00830885671036796\\
123.993	-0.00816698339515254\\
124.093	-0.00802514329584803\\
124.193	-0.00788333620556747\\
124.293	-0.00774156191904618\\
124.393	-0.00859982023262962\\
124.493	-0.00745811094426063\\
124.593	-0.00731643385346281\\
124.693	-0.00817478876133615\\
124.793	-0.00803317547053739\\
124.893	-0.00789159378527104\\
124.993	-0.0087500435112764\\
125.093	-0.00760852445581595\\
125.193	-0.00746703642766411\\
125.293	-0.0073255792370932\\
125.393	-0.0071841526958627\\
125.493	-0.0070427566172091\\
125.593	-0.00790139081583252\\
125.693	-0.00776005510788558\\
125.793	-0.00761874931096118\\
125.893	-0.0074774732440841\\
125.993	-0.00733622672769663\\
126.093	-0.00719500958364794\\
126.193	-0.00705382163518475\\
126.293	-0.00791266270693924\\
126.393	-0.00777153262491748\\
126.493	-0.00663043121649043\\
126.593	-0.00748935831038011\\
126.693	-0.00634831373665401\\
126.793	-0.00720729732670922\\
126.893	-0.00706630891326521\\
126.993	-0.00692534833035197\\
127.093	-0.00678441541330121\\
127.193	-0.00664350999873475\\
127.293	-0.00650263192455469\\
127.393	-0.00636178102993412\\
127.493	-0.00722095715530546\\
127.593	-0.00608016014235258\\
127.693	-0.00593938983399989\\
127.793	-0.00579864607440284\\
127.893	-0.00565792870893755\\
127.993	-0.00651723758419243\\
128.093	-0.00637657254795787\\
128.193	-0.00623593344921725\\
128.293	-0.00609532013813707\\
128.393	-0.0059547324660576\\
128.493	-0.00581417028548481\\
128.593	-0.00567363345007976\\
128.693	-0.00553312181464971\\
128.793	-0.00539263523514055\\
128.893	-0.00625217356862606\\
128.993	-0.00611173667330123\\
129.093	-0.00597132440847004\\
129.193	-0.00583093663453926\\
129.293	-0.0056905732130117\\
129.393	-0.00555023400647103\\
129.493	-0.00540991887858233\\
129.593	-0.00526962769407424\\
129.693	-0.00512936031873767\\
129.793	-0.00498911661941426\\
129.893	-0.00484889646398878\\
129.993	-0.00470869972138122\\
130.093	-0.00456852626153781\\
130.193	-0.00442837595542445\\
130.293	-0.00528824867501632\\
130.393	-0.00514814429329213\\
130.493	-0.00500806268422593\\
130.593	-0.00486800372277862\\
130.693	-0.00472796728488944\\
130.793	-0.00458795324747152\\
130.893	-0.00444796148839943\\
130.993	-0.00430799188650699\\
131.093	-0.00416804432157392\\
131.193	-0.00402811867432318\\
131.293	-0.00388821482641255\\
131.393	-0.00474833266042429\\
131.493	-0.00360847205986259\\
131.593	-0.00446863290914168\\
131.693	-0.00432881509358296\\
131.793	-0.0041890184994049\\
131.893	-0.00404924301371512\\
131.993	-0.00390948852450856\\
132.093	-0.00376975492065545\\
132.193	-0.00363004209189599\\
132.293	-0.00349034992883324\\
132.393	-0.0033506783229269\\
132.493	-0.00321102716648891\\
132.593	-0.00307139635267095\\
132.693	-0.00393178577546172\\
132.793	-0.00279219532968211\\
132.893	-0.00365262491097429\\
132.993	-0.00351307441579785\\
133.093	-0.00337354374142285\\
133.193	-0.00323403278592416\\
133.293	-0.00309454144817378\\
133.393	-0.00295506962783554\\
133.493	-0.00281561722535795\\
133.593	-0.0026761841419698\\
133.693	-0.0025367702796717\\
133.793	-0.00239737554123254\\
133.893	-0.00225799983018149\\
133.993	-0.00311864305080256\\
134.093	-0.00297930510812927\\
134.193	-0.00183998590793788\\
134.293	-0.00270068535674195\\
134.393	-0.00256140336178756\\
134.493	-0.00242213983104467\\
134.593	-0.00228289467320542\\
134.693	-0.00214366779767472\\
134.793	-0.00200445911456804\\
134.893	-0.00186526853470204\\
134.993	-0.00172609596959283\\
135.093	-0.00258694133144877\\
135.193	-0.0024478045331624\\
135.293	-0.00230868548831165\\
135.393	-0.00216958411114776\\
135.493	-0.00203050031659391\\
135.593	-0.00189143402023806\\
135.693	-0.00175238513832898\\
135.793	-0.00161335358777093\\
135.893	-0.0014743392861174\\
135.993	-0.00133534215156583\\
136.093	-0.00119636210295582\\
136.193	-0.00105739905975888\\
136.293	-0.000918452942079817\\
136.393	-0.000779523670644711\\
136.493	-0.000640611166801808\\
136.593	-0.00150171535251298\\
136.693	-0.00136283615035149\\
136.793	-0.00122397348349557\\
136.893	-0.00108512727572352\\
136.993	-0.000946297451410949\\
137.093	-0.0008074839355241\\
137.193	-0.000668686653614525\\
137.293	-0.00152990553181764\\
137.393	-0.00139114049684608\\
137.493	-0.000252391475984215\\
137.593	-0.000113658397085548\\
137.693	2.50588114325545e-05\\
137.793	-0.000836239779407499\\
137.893	-0.00069755409913741\\
137.993	-0.000558884077840549\\
138.093	-0.000420229646144854\\
138.193	-0.000281590735223158\\
138.293	-0.000142967276784756\\
138.393	-4.35920307140236e-06\\
138.493	0.000134233553143126\\
138.593	0.000272811058561917\\
138.693	0.000411373379365365\\
138.793	-0.000450079418782057\\
138.893	-0.000311547270730017\\
138.993	-0.000173030111836336\\
139.093	-3.45278779647629e-05\\
139.193	0.000103959494521799\\
139.293	0.00124243206876118\\
139.393	0.000380889907399151\\
139.493	0.000519333072591444\\
139.593	0.000657761626009279\\
139.693	0.000796175628841045\\
139.793	0.000934575141798621\\
139.893	0.00107296022511871\\
139.993	0.00121133093856729\\
140.093	0.00134968734144492\\
140.193	0.00148802949258808\\
140.293	0.00162635745037409\\
140.393	0.00176467127272462\\
140.493	0.0009029710171089\\
140.593	0.00104125674054778\\
140.693	0.00117952849961656\\
140.793	0.00131778635044943\\
140.893	0.00145603034874098\\
140.993	0.0015942605497532\\
141.093	0.00173247700831469\\
141.193	0.00187067977882815\\
141.292	0.0020074870912179\\
141.392	0.0021456627826848\\
141.492	0.00128382494624502\\
141.592	0.00242197363462449\\
141.692	0.00256010890013991\\
141.792	0.00169823079469911\\
141.892	0.00183633936980376\\
141.992	0.00197443467655534\\
142.092	0.00211251676565594\\
142.192	0.00225058568741154\\
142.292	0.00238864149173734\\
142.392	0.00252668422815772\\
142.492	0.00266471394581069\\
142.592	0.00280273069345238\\
142.692	0.00294073451945742\\
142.792	0.00307872547182386\\
142.892	0.0032167035981745\\
142.992	0.00335466894576175\\
143.092	0.00249262156147045\\
143.192	0.00363056149181684\\
143.292	0.00276848878295821\\
143.392	0.00290640348068871\\
143.492	0.00304430563044811\\
143.592	0.00318219527731989\\
143.692	0.00332007246603805\\
143.792	0.0034579372409862\\
143.892	0.00359578964620333\\
143.992	0.00373362972538427\\
144.092	0.00387145752188411\\
144.192	0.00400927307871957\\
144.292	0.00414707643857293\\
144.392	0.0042848676437921\\
144.492	0.00442264673639725\\
144.592	0.00456041375808036\\
144.692	0.00469816875020834\\
144.792	0.00483591175382614\\
144.892	0.00397364280965951\\
144.992	0.00411136195811546\\
145.092	0.00424906923928825\\
145.192	0.00438676469295851\\
145.292	0.00452444835859733\\
145.392	0.00466212027537027\\
145.492	0.00479978048213425\\
145.592	0.00493742901744776\\
145.692	0.00507506591956597\\
145.792	0.0052126912264483\\
145.892	0.00535030497575795\\
145.992	0.00548790720486458\\
146.092	0.00562549795084744\\
146.192	0.00576307725049707\\
146.292	0.00590064514031852\\
146.392	0.00603820165653168\\
146.492	0.00617574683507582\\
146.592	0.00631328071160864\\
146.692	0.00545080332151304\\
146.792	0.0065883146998944\\
146.892	0.00672581488158608\\
146.992	0.00586330390115108\\
147.092	0.0060007817928831\\
147.192	0.0061382485908088\\
147.292	0.00627570432869007\\
147.392	0.00641314904002765\\
147.492	0.00655058275806031\\
147.592	0.00668800551576965\\
147.692	0.00682541734587971\\
147.792	0.00696281828086098\\
147.892	0.00610020835293223\\
147.992	0.00723758759406046\\
148.092	0.00637495603596427\\
148.192	0.00651231371011818\\
148.292	0.00664966064774841\\
148.392	0.00778699687984163\\
148.492	0.00692432243714292\\
148.592	0.00706163735015775\\
148.692	0.00719894164915624\\
148.792	0.00733623536417216\\
148.892	0.00747351852500566\\
148.992	0.00761079116122731\\
149.092	0.0077480533021772\\
149.192	0.00788530497696671\\
149.292	0.00802254621448251\\
149.392	0.00815977704338655\\
149.492	0.00829699749211876\\
149.592	0.00843420758889657\\
149.692	0.00857140736172202\\
149.792	0.00770859683837655\\
149.892	0.00884577604642711\\
149.992	0.00898294501322772\\
150.092	0.00812010376591932\\
150.192	0.00825725233143171\\
150.292	0.00839439073648762\\
150.392	0.00853151900760141\\
150.492	0.00866863717108179\\
150.592	0.00880574525303413\\
150.692	0.0089428432793599\\
150.792	0.00907993127576301\\
150.892	0.00921700926774482\\
150.992	0.00935407728061044\\
151.092	0.00949113533946955\\
151.192	0.00962818346923511\\
151.292	0.00976522169463045\\
151.392	0.00890225004018408\\
151.492	0.00903926853023629\\
151.592	0.00917627718893854\\
151.692	0.00931327604025567\\
151.792	0.00945026510796554\\
151.892	0.0095872444156635\\
151.992	0.00972421398676149\\
152.092	0.00986117384449114\\
152.192	0.00999812401190248\\
152.292	0.0101350645118683\\
152.392	0.0102719953670851\\
152.492	0.0104089166000727\\
152.592	0.0105458282331763\\
152.692	0.0106827302885697\\
152.792	0.0108196227882544\\
152.892	0.0109565057540619\\
152.992	0.010093379207655\\
153.092	0.0102302431705286\\
153.192	0.0103670976640111\\
153.292	0.0105039427092679\\
153.392	0.0106407783272986\\
153.492	0.0107776045389425\\
153.592	0.010914421364876\\
153.692	0.0110512288256164\\
153.792	0.0111880269415243\\
153.892	0.0113248157328001\\
153.992	0.0114615952194903\\
154.092	0.0115983654214857\\
154.192	0.0117351263585244\\
154.292	0.0118718780501905\\
154.392	0.0110086205159186\\
154.492	0.0121453537749932\\
154.592	0.0112820778465483\\
154.692	0.0114187927495717\\
154.792	0.011555498502906\\
154.892	0.0116921951252449\\
154.992	0.0118288826351418\\
155.092	0.0119655610510043\\
155.192	0.013102230391099\\
155.292	0.0122388906735522\\
155.392	0.0123755419163487\\
155.492	0.0125121841373379\\
155.592	0.0116488173542293\\
155.692	0.0127854415845956\\
155.792	0.0129220568458752\\
155.892	0.0130586631553724\\
155.992	0.0131952605302579\\
156.092	0.01333184898757\\
156.192	0.0134684285442157\\
156.292	0.0126049992169728\\
156.392	0.012741561022489\\
156.492	0.0128781139772833\\
156.592	0.0130146580977502\\
156.692	0.0131511934001556\\
156.792	0.0132877199006409\\
156.892	0.013424237615224\\
156.992	0.0135607465597993\\
157.092	0.0136972467501391\\
157.192	0.0138337382018938\\
157.292	0.0139702209305947\\
157.392	0.0131066949516532\\
157.492	0.0142431602803614\\
157.592	0.0143796169318957\\
157.692	0.0145160649213154\\
157.792	0.0146525042635632\\
157.892	0.0147889349734673\\
157.992	0.013925357065744\\
158.092	0.0140617705549948\\
158.192	0.0141981754557086\\
158.292	0.0143345717822654\\
158.392	0.0144709595489334\\
158.492	0.0146073387698706\\
158.592	0.0147437094591285\\
158.692	0.0148800716306505\\
158.792	0.0150164252982714\\
158.892	0.0151527704757215\\
158.992	0.0152891071766259\\
159.092	0.0154254354145045\\
159.192	0.0155617552027749\\
159.292	0.015698066554751\\
159.392	0.0158343694836445\\
159.492	0.0149706640025671\\
159.592	0.0151069501245291\\
159.692	0.0152432278624417\\
159.792	0.016379497229118\\
159.892	0.0155157582372722\\
159.992	0.015652010899522\\
160.092	0.0157882552283883\\
160.192	0.0159244912362948\\
160.292	0.0160607189355733\\
160.392	0.0161969383384593\\
160.492	0.016333149457096\\
160.592	0.016469352303532\\
160.692	0.0156055468897263\\
160.792	0.0167417332275455\\
160.891	0.0168765495884853\\
160.991	0.0170127195469836\\
161.091	0.0171488812920502\\
161.191	0.0172850348351945\\
161.291	0.0164211801878387\\
161.391	0.0165573173613169\\
161.491	0.0166934463668786\\
161.591	0.016829567215686\\
161.691	0.0169656799188158\\
161.791	0.0171017844872625\\
161.891	0.0172378809319356\\
161.991	0.0173739692636605\\
162.091	0.017510049493183\\
162.191	0.0176461216311639\\
162.291	0.0177821856881861\\
162.391	0.017918241674749\\
162.491	0.0180542896012748\\
162.591	0.0181903294781049\\
162.691	0.0183263613155025\\
162.791	0.0184623851236525\\
162.891	0.0175984009126644\\
162.991	0.0187344086925672\\
163.091	0.0178704084733181\\
163.191	0.0190064002647952\\
163.291	0.0181423840768038\\
163.391	0.0192783599190731\\
163.491	0.0184143278012612\\
163.591	0.0195502877329501\\
163.691	0.0186862397236518\\
163.791	0.0188221837828029\\
163.891	0.0189581199197719\\
163.991	0.0190940481438546\\
164.091	0.0192299684642769\\
164.191	0.0193658808901946\\
164.291	0.0195017854306947\\
164.391	0.0196376820947948\\
164.491	0.019773570891445\\
164.591	0.0189094518295274\\
164.691	0.0200453249178567\\
164.791	0.0201811901651823\\
164.891	0.019317047580186\\
164.991	0.0204528971714848\\
165.091	0.0195887389476304\\
165.191	0.0197245729171094\\
165.291	0.0198603990883464\\
165.391	0.0199962174696999\\
165.491	0.0201320280694679\\
165.591	0.0202678308958837\\
165.691	0.0204036259571194\\
165.791	0.0205394132612864\\
165.891	0.0206751928164342\\
165.991	0.0208109646305514\\
166.091	0.020946728711567\\
166.191	0.0210824850673506\\
166.291	0.0212182337057114\\
166.391	0.021353974634402\\
166.491	0.020489707861115\\
166.591	0.0216254333934853\\
166.691	0.0217611512390921\\
166.791	0.0218968614054553\\
166.891	0.0220325639000412\\
166.991	0.0221682587302574\\
167.091	0.0213039459034574\\
167.191	0.022439625426939\\
167.291	0.0215752973079466\\
167.391	0.0227109615536687\\
167.491	0.0218466181712405\\
167.591	0.0229822671677442\\
167.691	0.022117908550209\\
167.791	0.0222535423256103\\
167.891	0.0223891685008737\\
167.991	0.0225247870828711\\
168.091	0.0226603980784237\\
168.191	0.0227960014943016\\
168.291	0.0229315973372244\\
168.391	0.0230671856138622\\
168.491	0.0232027663308347\\
168.591	0.0233383394947118\\
168.691	0.0234739051120156\\
168.791	0.0236094631892185\\
168.891	0.0237450137327455\\
};
\end{axis}
\end{tikzpicture}%
    \caption{Residual of $V = A\exp(Bt) + C\exp(Dt)$
      fit.\label{fig:complex_resid}}
  \end{subfigure}
  \caption{Double Exponential Fit: Data from \SI{400}{\nano\meter}
    trial\label{fig:complex}}
\end{figure}

Fig.~\ref{fig:evsf} shows the energy of the light at different frequencies, and
has a slope of $h$. Our experimental value of $h = \num{6.2781 +- .12064 e-34}$
was close to the theoretical value, with only \SI{6.04}{\percent} error. Perhaps
voltage leakages from the capacitor caused a divergence from the ideal RC
circuit that our fit didn't capture, yielding some error in the fit's
approximation and in the calculation of Planck's constant. We assumed that the
voltage drop across the LED was constant, which could also have resulted in poor
fits. In addition, a definitive source of error arises from relying on the
wavelengths of the LEDs reported by the manufacturer. 

\begin{figure}[h]
  \centering
  % This file was created by matlab2tikz.
%
\definecolor{mycolor1}{rgb}{0.00000,0.44700,0.74100}%
\definecolor{mycolor2}{rgb}{0.85000,0.32500,0.09800}%
%
\begin{tikzpicture}[%
trim axis left
]

\begin{axis}[%
width=1.902in,
height=2in,
at={(0in,0in)},
scale only axis,
xmin=400000000000000,
xmax=800000000000000,
xlabel style={font=\color{white!15!black}},
xlabel={Frequency (\si{\hertz})},
ymin=2.5e-19,
ymax=5.5e-19,
ylabel style={font=\color{white!15!black}},
ylabel={Energy (\si{\joule})},
axis background/.style={fill=white},
axis x line*=bottom,
axis y line*=left,
legend style={at={(0.03,0.97)}, anchor=north west, legend cell align=left, align=left, draw=white!15!black},
legend style={font=\small}
]
\addplot[only marks, mark=o, mark options={}, mark size=1.5000pt, draw=mycolor1] table[row sep=crcr]{%
x	y\\
778681709090909	4.81103771692778e-19\\
758968248101266	4.81912182172225e-19\\
749481145000000	4.72791125275254e-19\\
740228291358025	4.6507633297149e-19\\
697191762790698	4.34356926725962e-19\\
643331454935622	5.0356978811933e-19\\
540166590990991	3.09850334114791e-19\\
530606120353982	3.16649747259692e-19\\
525951680701754	3.12233410258083e-19\\
521378187826087	3.13005811279515e-19\\
512465740170940	3.031170480764e-19\\
508122810169491	3.02819573161254e-19\\
504701107744108	2.99793483937841e-19\\
499654096666667	2.93112412543666e-19\\
493079700657895	2.93066356120829e-19\\
475861044444444	2.98027898146227e-19\\
472114107086614	2.86803807474789e-19\\
466240214618974	2.81148793630728e-19\\
461219166153846	3.19708308401669e-19\\
428274940000000	3.17351977588301e-19\\
};
\addlegendentry{Experimental Data}

\addplot [color=mycolor2, line width=2.0pt]
  table[row sep=crcr]{%
428274940000000	2.68873524863785e-19\\
778681709090909	4.88860954297792e-19\\
};
\addlegendentry{Fit}

\end{axis}
\end{tikzpicture}%
  \caption{Energy vs. Frequency: Stopping voltages are the constant $C$ obtained
    from the fits to the voltage data and wavelengths are the manufacturer
    reported wavelengths of the LEDs. Fit is linear with $r^2 = 0.8664$.\label{fig:evsf}}
\end{figure}

Fig.~\ref{fig:sun} shows the blackbody radiation curve generated by our data.
From the fit, we computed $h =\num{7.7868 +- .60495 e-34}$. This is within
\SI{17.52}{\percent} of the accepted value, which is better than we had hoped
for a method with so many possible sources of noise. Possible sources of error
include imperfect blocking of external light, imprecise orientation of the
filter resulting in reduced light intensity, errors in correcting for the
spectral absorption of the photometer and atmospheric absorption, and deviations
in the sun's true radiation spectrum from an ideal blackbody at
\SI{5778}{\kelvin}.

\begin{figure}[h]
  \centering
  % This file was created by matlab2tikz.
%
\definecolor{mycolor1}{rgb}{0.00000,0.44700,0.74100}%
\definecolor{mycolor2}{rgb}{0.85000,0.32500,0.09800}%
%
\begin{tikzpicture}[%
trim axis left
]

\begin{axis}[%
width=1.902in,
height=2in,
at={(0in,0in)},
scale only axis,
xmin=0,
xmax=1500,
xlabel style={font=\color{white!15!black}},
xlabel={Wavelength (\si{\nano\meter})},
ymin=0,
ymax=0.7,
ylabel style={font=\color{white!15!black}},
ylabel={Intensity (arb. units)},
axis background/.style={fill=white},
axis x line*=bottom,
axis y line*=left,
legend style={at={(0.97,0.03)}, anchor=south east, legend cell align=left, align=left, draw=white!15!black},
legend style={font=\small}
]
\addplot[only marks, mark=o, mark options={}, mark size=1.5000pt, draw=mycolor1] table[row sep=crcr]{%
x	y\\
480	0.604082270316182\\
500	0.591706995992351\\
620	0.527128076562626\\
360	0.200686823489145\\
500	0.557619155683626\\
540	0.658478735104653\\
685	0.542569678139747\\
};
\addlegendentry{Corrected Data}

\addplot [color=mycolor2, line width=2.0pt]
  table[row sep=crcr]{%
1.5015015015015	0\\
145.645645645646	0.000174944270021209\\
159.159159159159	0.000618153001905889\\
168.168168168168	0.00125689108558946\\
175.675675675676	0.00212526463184387\\
183.183183183183	0.00341213386754789\\
189.189189189189	0.00482188598746503\\
195.195195195195	0.00663810113336562\\
201.201201201201	0.00892458163859195\\
205.705705705706	0.0109856233654852\\
210.21021021021	0.0133713500409998\\
214.714714714715	0.0161052860094354\\
219.219219219219	0.019209022347159\\
223.723723723724	0.0227018722335396\\
228.228228228228	0.0266005802430232\\
232.732732732733	0.0309190890405145\\
237.237237237237	0.0356683648293288\\
241.741741741742	0.0408562810281258\\
246.246246246246	0.0464875580739517\\
250.750750750751	0.0525637559747578\\
255.255255255255	0.0590833152614995\\
259.75975975976	0.066041641299486\\
264.264264264264	0.0734312264847212\\
268.768768768769	0.0812418046419854\\
273.273273273273	0.0894605319232814\\
277.777777777778	0.0980721886436482\\
282.282282282282	0.107059396753181\\
286.786786786787	0.116402847998758\\
291.291291291291	0.126081538248956\\
297.297297297297	0.139468934697922\\
303.303303303303	0.153355708296681\\
309.309309309309	0.167682424745296\\
315.315315315315	0.182387662778727\\
322.822822822823	0.201206177903953\\
331.831831831832	0.224267377742215\\
343.843843843844	0.255486243989349\\
363.363363363363	0.306244176425496\\
372.372372372372	0.329255737471202\\
379.87987987988	0.348069511807525\\
387.387387387387	0.366467929144668\\
393.393393393393	0.380839201533016\\
399.399399399399	0.394865864339098\\
405.405405405405	0.408518529269446\\
411.411411411411	0.421771138470482\\
417.417417417417	0.434600869240267\\
423.423423423423	0.446988018765089\\
429.429429429429	0.45891587296883\\
435.435435435435	0.47037056315332\\
441.441441441441	0.481340913705677\\
447.447447447447	0.491818283761009\\
453.453453453453	0.501796405341069\\
459.459459459459	0.511271220144975\\
465.465465465465	0.520240716849198\\
471.471471471471	0.52870477048174\\
477.477477477477	0.536664985170033\\
483.483483483483	0.544124541323294\\
489.489489489489	0.551088048096832\\
495.495495495496	0.557561401797109\\
501.501501501501	0.563551650720563\\
507.507507507508	0.569066866774779\\
513.513513513514	0.574116024105911\\
519.519519519519	0.578708884849455\\
525.525525525525	0.582855892030986\\
531.531531531532	0.586568069567646\\
537.537537537538	0.589856929258347\\
543.543543543544	0.592734384599472\\
549.54954954955	0.595212671221873\\
555.555555555556	0.597304273712932\\
561.561561561562	0.599021858563176\\
567.567567567568	0.600378212959418\\
573.573573573574	0.601386189134584\\
579.57957957958	0.602058653977536\\
587.087087087087	0.602446970965003\\
594.594594594595	0.602356005909804\\
602.102102102102	0.601810378811058\\
609.60960960961	0.600834332115316\\
617.117117117117	0.599451650374695\\
624.624624624625	0.597685592596801\\
632.132132132132	0.595558836083032\\
641.141141141141	0.592561605632692\\
650.15015015015	0.589113519941345\\
659.159159159159	0.585249969208784\\
668.168168168168	0.581004852569545\\
678.678678678679	0.575613072421041\\
689.189189189189	0.569794730934377\\
699.6996996997	0.563595778639709\\
711.711711711712	0.556100434937602\\
723.723723723724	0.548224264324046\\
737.237237237237	0.538977285828524\\
752.252252252252	0.528307657925275\\
770.27027027027	0.515076408249831\\
791.291291291291	0.499221681599916\\
821.321321321321	0.47613791811379\\
878.378378378378	0.432229798611457\\
905.405405405405	0.411891681573406\\
927.927927927928	0.395326021501238\\
948.948948948949	0.380237568766097\\
969.96996996997	0.365547918394843\\
989.48948948949	0.352289117639355\\
1009.00900900901	0.339414181968605\\
1028.52852852853	0.326933785227366\\
1048.04804804805	0.314854214210567\\
1067.56756756757	0.303178084290728\\
1087.08708708709	0.291904959235173\\
1106.60660660661	0.281031886247975\\
1126.12612612613	0.270553856345586\\
1145.64564564565	0.260464199231071\\
1165.16516516517	0.250754920906904\\
1186.18618618619	0.240713793136346\\
1207.20720720721	0.231091035440132\\
1228.22822822823	0.221873584962409\\
1249.24924924925	0.21304785804952\\
1271.77177177177	0.204010620133456\\
1294.29429429429	0.195389828895737\\
1316.81681681682	0.187168203895389\\
1339.33933933934	0.179328615606795\\
1363.36336336336	0.171368490330378\\
1387.38738738739	0.163803865514449\\
1412.91291291291	0.156178084729075\\
1438.43843843844	0.148954358069558\\
1465.46546546547	0.141719914861657\\
1492.49249249249	0.134887444000752\\
1500	0.133057697008513\\
};
\addlegendentry{Fit}

\end{axis}
\end{tikzpicture}%
  \caption{Solar Emission Spectrum: \textit{Corrected Data} shows the intensity data
    after corrections for the photodiode spectral response and atmospheric
    absorption. The fit represents a fit to the shape of the curve in Planck's
    law, $I(\lambda) = A\lambda^{-5} {\qty(\exp(B \lambda^{-1} -1))}^{-1} $ and has $r^2 =
    0.8152$.\label{fig:sun}}
\end{figure}

Table~\ref{tab:h} compares the measured values of Planck's constant from both
methods with the accepted value. We can see that the LED method was more
accurate, which is consistent with our expectations: the LED method is simpler
than the solar radiation method because relies on our ability to model
electronics in the lab, while the solar radiation depends on our ability to
model the sun, the atmosphere, and the photodiode.

\begin{table}[h]
  \centering
  \begin{tabular}{ccc}
    \toprule
        & $h$ (\si{\joule\second})    & Percent Error        \\
    \midrule
    LED & \num{6.2781 +- .12064 e-34} & \SI{6.04}{\percent}  \\
    Sun & \num{7.7868 +- .60495 e-34} & \SI{17.52}{\percent} \\
    \bottomrule
  \end{tabular}
  \caption{Computed Values for Planck's Constant\label{tab:h}}
\end{table}

\section{Conclusion}
\label{Sec:Conc}

Inherent in both the electron and in the sun, in the atomic and in the
celestial, can be found one of the fundamental constants of the universe:
Planck's constant. We used two methods of experimentally determining this
constant that were related to two of the earliest successes of the quantum
theory. The LED method proved to be more successful than the blackbody method,
which we ascribe to the controlled environment in which it was performed; the
blackbody experiment had many more uncontrolled variables. We consider it a
significant success that that in both experiments the error was below
\SI{20}{\percent}. To improve our results for the LED method, we would first
measure the frequencies of the LEDs directly, since the frequencies listed by
the manufacturer as the frequencies of the LEDs themselves have errors. In
addition, we would try to correct for any voltage leakage in the capacitor and
LED to more nearly approximate an ideal RC-circuit. To improve our results for
the blackbody experiment, we would build an apparatus that would secure the
photometer and the filter to point in the same angle so that all the light
transmitted through the filter would hit the photometer directly.

\section{Acknowledgments}
\label{Sec:Acks}
Thanks to Jenn for obtaining so many LEDs, for allowing us to use the filters in
the filter cubes. Thanks also to Quan for providing us with more filters from
his lab and for discussing how to deal with leakage in the capacitor.

\pledge{Jake Waksbaum and Oriel Farajun}

\nocite{*}
\printbibliography%
\label{Sec:Refs}

\section{Supplementary Information}
\label{Sec:SupInfo}

\lstinputlisting[language=Matlab]{matlab/sun/sun.m}
\lstinputlisting[language=Matlab]{matlab/led/calcLEDH.m}
\lstinputlisting[language=Matlab]{matlab/led/getVF.m}
\lstinputlisting[language=Matlab]{matlab/led/leds.m}
\lstinputlisting[language=Matlab]{matlab/led/vfTable.m}
\lstinputlisting[language=Matlab]{matlab/lib/parseDateISO.m}
\lstinputlisting[language=Matlab]{matlab/lib/Constants.m}
\lstinputlisting[language=Matlab]{matlab/sun/airMass.m}
\lstinputlisting[language=Matlab]{matlab/sun/calcSunH.m}
\lstinputlisting[language=Matlab]{matlab/sun/correctAtmosphere.m}
\lstinputlisting[language=Matlab]{matlab/sun/findSpecResponse.m}
\lstinputlisting[language=Matlab]{matlab/sun/getC13.m}

\end{document}
